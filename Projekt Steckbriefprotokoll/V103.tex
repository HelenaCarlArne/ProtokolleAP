\documentclass[
    parskip=half,
    bibliography=totoc,     % Literatur im Inhaltsverzeichnis
    captions=tableheading,  % Tabellen�berschriften
    titlepage=firstiscover, % Titelseite ist Deckblatt
    ]{scrartcl}
    
\usepackage[top=2cm, bottom=4cm, left=2cm, right=2cm]{geometry}
\usepackage{color}
\usepackage[usenames,dvipsnames]{xcolor}
\definecolor{light-red}{HTML}{FFBABA}

% LaTeX2e korrigieren.
\usepackage{fixltx2e}

% Warnung, falls nochmal kompiliert werden muss
\usepackage[aux]{rerunfilecheck}

% Deutsche Spracheinstellungen
\usepackage{polyglossia}
\setmainlanguage{german}

% Unverzichtbare Mathe-Befehle
\usepackage{amsmath}

% Viele Mathe-Symbole
\usepackage{amssymb}

% Erweiterungen f�r amsmath
\usepackage{mathtools}

% Fonteinstellungen
\usepackage{fontspec}
\defaultfontfeatures{Ligatures=TeX}

\usepackage[
    math-style=ISO,    % \
    bold-style=ISO,    % |
    sans-style=italic, % | ISO-Standard folgen
    nabla=upright,     % |
    partial=upright,   % /
    ]{unicode-math}

\setmathfont{Latin Modern Math}
\setmathfont[range={\mathscr, \mathbfscr}]{XITS Math}
\setmathfont[range=\coloneq]{XITS Math}
\setmathfont[range=\propto]{XITS Math}

% Das hquer-Symbol versch�nern
\let\hbar\relax
\DeclareMathSymbol{\hbar}{\mathord}{AMSb}{"7E}
\DeclareMathSymbol{?}{\mathord}{AMSb}{"7E}

% Richtige Anf�hrungszeichen
\usepackage[autostyle]{csquotes}

% Zahlen und Einheiten
\usepackage[
  locale=DE,                   % Deutsche Einstellungen
  separate-uncertainty=true,   % Immer Fehler mit \pm
  per-mode=symbol-or-fraction, % m/s im Text, sonst Br�che
]{siunitx}

% Chemische Formeln
\usepackage[version=3]{mhchem}

% Sch�ne Br�che im Text
\usepackage{xfrac}

% Floats innerhalb einer Section halten
\usepackage[section, below]{placeins}

% Captions sch�ner machen.
\usepackage[
    labelfont=bf,        % Tabelle x: Abbildung y: ist jetzt fett
    font=small,          % Schrift etwas kleiner als Dokument
    width=0.9\textwidth, % Maximale Breite einer Caption schmaler
    ]{caption}

% Subfigure, subtable, subref
\usepackage{subcaption}

% Grafiken einbinden
\usepackage{graphicx}

% Gr��ere Variation von Dateinamen m�glich
\usepackage{grffile}

% Standardplatzierung f�r Floats einstellen
\usepackage{float}
\floatplacement{figure}{htbp}
\floatplacement{table}{htbp}

% Sch�ne Tabellen
\usepackage{booktabs}

% Seite drehen f�r breite Tabellen
\usepackage{pdflscape}

% Literaturverzeichnis
\usepackage{biblatex}

% Quellendatenbank
\addbibresource{lit.bib}
\addbibresource{programme.bib}

% Hyperlinks im Dokument
\usepackage[
    unicode,
    pdfusetitle,    % Titel, Autoren und Datum als PDF-Attribute
    pdfcreator={},  % PDF-Attribute s�ubern
    pdfproducer={}, % "
    ]{hyperref}
    
% Erweiterte Bookmarks im PDF
\usepackage{bookmark}

% Trennung von W�rtern mit Strichen
\usepackage[shortcuts]{extdash}

% Blindtext erzeugen
\usepackage{blindtext}

% Support f�r mdframed
\usepackage{kvoptions}
\usepackage{xparse}
\usepackage{etoolbox}
\usepackage{tikz}

% Sch�ne mehrseitige Rahmen um Text erzeugen
\usepackage{mdframed}
\mdfsetup{skipabove=\topskip,skipbelow=\topskip}
% Neue Mathematikbefehle
\DeclareMathOperator{\rank}{rang}
\DeclareMathOperator{\cond}{cond}
\newcommand{\up}{\mathup}
\newcommand{\R}{\mathbb{R}}
\newcommand{\N}{\mathbb{N}}
\newcommand{\upD}{\mathup{\Delta}}

% Wichtiger Befehl zur Erstellung einer blauen Box um einen Text
\newcommand\mybox[2][]{\tikz[overlay]\node[fill=blue!20,inner sep=4pt, anchor=text, rectangle, rounded corners=1mm,#1] {#2};\phantom{#2}}
\newenvironment{Versuch}[1]{
    \mdfsetup{
        innertopmargin=8pt,
        linecolor=blue!20,
        linewidth=2pt,
        topline=true,
        backgroundcolor=blue!20
        }
    \begin{mdframed}
    \Large{\textbf{#1}}
    \end{mdframed}}
    {}

\newenvironment{Stichworte}{
    \mdfsetup{
        frametitle={\mybox[fill=blue!20]{\Large{Stichworte}}},
        frametitleaboveskip=0pt,
        innertopmargin=10pt,
        linecolor=blue!20,
        linewidth=2pt,
        topline=true,
        backgroundcolor=white
        }
    \begin{mdframed}}
    {\end{mdframed}}    
    
\newenvironment{Zielsetzung}{
    \mdfsetup{
        frametitle={\mybox[fill=blue!20]{\Large{Zielsetzung}}},
        frametitleaboveskip=0pt,
        innertopmargin=10pt,
        linecolor=blue!20,
        linewidth=2pt,
        topline=true,
        backgroundcolor=white
        }
    \begin{mdframed}}
    {\end{mdframed}}

\newenvironment{Theorie}{
    \mdfsetup{
        frametitle={\mybox[fill=blue!20]{\Large{Theorie}}},
        frametitleaboveskip=0pt,
        innertopmargin=10pt,
        linecolor=blue!20,
        linewidth=2pt,
        topline=true,
        backgroundcolor=white
        }
    \begin{mdframed}}
    {\end{mdframed}}

\newenvironment{Durchführung}{
    \mdfsetup{
        frametitle={\mybox[fill=blue!20]{\Large{Durchführung}}},
        frametitleaboveskip=0pt,
        innertopmargin=10pt,
        linecolor=blue!20,
        linewidth=2pt,
        topline=true,
        backgroundcolor=white
        }
    \begin{mdframed}}
    {\end{mdframed}}
    
\newenvironment{Auswertung}{
    \mdfsetup{
        frametitle={\mybox[fill=blue!20]{\Large{Auswertung}}},
        frametitleaboveskip=0pt,
        innertopmargin=10pt,
        linecolor=blue!20,
        linewidth=2pt,
        topline=true,
        backgroundcolor=white
        }
    \begin{mdframed}}
    {\end{mdframed}}
    
\newenvironment{Diskussion}{
    \mdfsetup{
        frametitle={\mybox[fill=blue!20]{\Large{Diskussion}}},
        frametitleaboveskip=0pt,
        innertopmargin=10pt,
        linecolor=blue!20,
        linewidth=2pt,
        topline=true,
        backgroundcolor=white
        }
    \begin{mdframed}}
    {\end{mdframed}}

\newenvironment{Merke}{
    \mdfsetup{
        frametitle={\mybox[fill=red]{\Large{Merke}}},
        frametitleaboveskip=0pt,
        innertopmargin=5pt,
        linecolor=red,
        linewidth=1pt,
        topline=true,
        backgroundcolor=light-red
        }
    \begin{mdframed}}
    {\end{mdframed}}

\newenvironment{Appendix}{
    \mdfsetup{
        frametitle={\mybox[fill=blue!20]{\Large{Appendix}}},
        frametitleaboveskip=0pt,
        innertopmargin=10pt,
        linecolor=blue!20,
        linewidth=0pt,
        topline=false,
        backgroundcolor=white
        }
    \begin{mdframed}}
    {\end{mdframed}}
    
\begin{document}

    \begin{Versuch}{V103: Biegung elastischer Stäbe}
    	\begin{Stichworte}
    		Hooksches Gesetz (Federn und Allgemein), Tangential- und Normalkomponente der Spannung $\sigma$, neutrale Faser, Drehmoment, Flächenträgheitsmoment, innere Spannung oder Gegenspannung
    	\end{Stichworte}

      \begin{Zielsetzung}
        Es werden die Kräfte kennengelernt, die beim Biegen von Stäben auftreten. Weiter werden Materialkonstanten eingeführt (hierzu weiterführend: Drehschwingungen)
      \end{Zielsetzung}

        \begin{Theorie}
            Das Experiment liegt in zwei Varianten vor: 
            einseitige und beidseitige Einspannung. 
            Die Varianten sind ähnlich und lassen einen Vergleich bezüglich Handlichkeit und Experimentatorgeschick zu.

            Das Hooke'sche Gesetz beschreibt in seiner allgemeinen Form 
            (neben der Schulvariante von Federn)
            die relative Längen- und Gestaltänderungen $\frac{\upD L}{L}$ von Proben, 
            worauf kleine, nicht-brechende Kräfte wirken.
            \begin{equation}
            	\sigma = E\cdot\frac{\upD L}{L}
            \end{equation}
            Anstelle von Kräften auf Oberflächen wird von mech.
            Spannung $\sigma$ gesprochen,
            welches als Zugkraft pro Querschnittfläche definiert ist.
            \begin{equation}
            	\sigma=\frac {|\vec {F}|}{A}
            \end{equation}
            Druck ist ein Spezialfall dieser Spannung.
            Spannung wird unterteilt in tangentialer (in Zugkraftrichtung) und normaler (senkrecht zur Zugkraftrichtung) Komponente.
            Normalkräfte verjüngen oder erweiten Probendurchmesser, sie werden ab jetzt nicht weiter berücksichtigt.

            Tangentiale Spannungen im Stab lassen sich veranschaulichen,
            indem man sie als senkrechtstehende Kraft auf die Probenquerschnittfläche annimmt.
            Einfach gekrümmte Stäbe besitzen eine neutrale Faser, 
            die weder in Länge gestaucht oder gestreckt wird.
            Nach den obigen Annahmen steht eine neutrale Faser nicht unter Spannung.
            Die Krümmung eines Stabes rührt daher, dass wegen der Spannung der Stabanteil oberhalb der neutralen Faser gestreckt und der untere Teil gestaucht wird.

            Es ist eine Materialeigenschaft von Stoffen,
            verschieden auf biegende äußere Spannung zu reagieren,
            haptisch sind manche leicht biegbar andere starr.
            Diese Eigenschaft lässt sich verdeutlichen, 
            indem die innere Spannung oder Gegenspannung betrachtet wird.
            Ein unter Last biegende Stab erfährt von außen ein biegendes Moment.
            Diesem wirkt die sich aufbauende innere Spannung des Stabes entgegen, sodass ein Kräftegleichgewicht entsteht und der Stab seine Krümmung nicht weiter erhöht.
           \begin{equation}
           	M_\text{außen}=\mkern-28mu\underbrace{F(L-x)}_{\begin{tabular}{c}\scriptsize äußeres Moment:\\ \scriptsize Kraft mal Hebelarm\end{tabular}}\mkern-28mu=\underbrace{\int\limits_{\text{Querschnitt}}\mkern-28mu\sigma \text{d}q}_{\begin{tabular}{c} \scriptsize innere Spannung,\\ \scriptsize Gegenspannung\end{tabular}}\mkern-16mu=M_\text{innen}
           \end{equation}
           Mit differentialgeometrischer Annährung $\frac{1}{R}=\frac{\text{d}^2 D(x)}{\text{d}x^2}$, wobei $R$ Krümmungsradius und $D(x)$ die Deformation bei $x$, also die Auslenkung von Ruhelage, ist, folgt nach zweifacher Integration und Anwenden der Randbedingungen
           \begin{equation}
           		D(x)=\frac{F}{2EI}(Lx^2-\frac{1}{3}x^3)\qquad\text{für}\qquad 0\le x\le L
           		\label{def1}
           \end{equation}
           Darin ist 
           \begin{equation}
           	I= \mkern-28mu \int\limits_{\text{Querschnitt}}\mkern-28mu y^2 \text{d}q
           \end{equation}
           das sogenannte Flächenträgheitsmoment 
           (wegen formaler Analogie zum Massenträgheitsmoment),
           welches eine Stabform abhängige Konstante ist.
           $y$ ist die Parametrisierung des Querschnitts; bei $y=0$ liegt die neutrale Faser, der Rand ist bei $y=r$ erreicht.

           Bei beidseitiger Einspannung ändert sich die Randbedingung und man erhält
           \begin{alignat}{3}
           		D(x)&=\frac{F}{48EI}(3L^2x-4x^3)\qquad&&\text{für}\qquad 0\le x\le \frac{L}{2}\\
           		D(x)&=\frac{F}{48EI}(4x^3-12Lx^2+9^2x-L^3)\qquad&&\text{für}\qquad \frac{L}{2}\le x\le L
           		\label{def2}
           \end{alignat}
        \end{Theorie}
        
        \begin{Durchführung}
            \begin{itemize}
            \item Verschiedene Stäbe auswählen und Abmessung mehrfach bestimmen.
            \item Stab in Vorrichtung einspannen, die die Deformation anhand einer auf Schiene gelagerten Messuhr bestimmen kann.
            Schiene und unbelasteter Stab liegen parallel.
            \item Gewicht an freies Stabende respektive Stabmitte anhängen und fixieren.
            \item In gewählten Abständen $x$ Messungen der Deformation $D$ durchführen, Offset ist der unbelastete Stab.
            \end{itemize}
        \end{Durchführung}

        \begin{Auswertung}
	        \begin{itemize}
	        	\item Es wird die Deformation gegen den Klammerausdruck in \eqref{def1} oder \eqref{def2} aufgetragen.
	        	Durch diese Linearisierung wird die Steigung regrediert und mithilfe dieser den Elastizitätsmodul $E$ bestimmt.
	        	\item Bei beidseitiger Einspannung reicht es konzeptionell aus, nur eine der Seiten vom Gewicht ausgehend zu messen.
	        	Durch Messung beider Seiten ist die Messfehlersicherheit erhöht.
	        	\item Das Flächenträgheitsmodul hängt nur von der Geometrie des Stabes ab und wird vor der Regression für die verwendeten Stäbe errechnet.
	        	\item Der Elastizitätsmodul ist ein großer Wert:
	        	Baustahl: $\SI{210}{\giga\pascal}$, Alu: $\SI{70}{\giga\pascal}$
        	\end{itemize}
        \end{Auswertung}
        
        \begin{Diskussion}
	        \begin{description}
	        	\item [Fehleranfälligkeit] Es konnten im Bereich von $5\%$ präzise gemessen werden.
	        	\item [Verfahrenwahl] Einseitig ist vorteilhaft für die größeren und damit leichter zu bestimmenden Werte der Deformation.
	        	Zweiseitig ist vorteilhaft für den zusätzlichen redundanten Datensatz (beide Seiten vom Gewicht).
        	\end{description}
        \end{Diskussion}

        \begin{Merke}
        	\begin{itemize}
				\item Die (Normal-)spannung $\sigma$ sind Kräfte, die senkrecht zur Fläche wirken. 
				Sie werden je nach Richtung Zugspannung (positives Vorzeichen) oder Druckspannung (negatives Vorzeichen) genannt. 
				Druck ist also ein Spezialfall der Spannung.
				\item Die tangentiale Spannungen $\tau$ werden als Schubspannungen bezeichnet. 
				Sie wirken tangential zur Fläche, stellen also eine Belastung auf Scherung dar.
				\item Bei mechanischen Experimenten mit Verklemmungen rechnen: abklopfen!
        	\end{itemize}
		\end{Merke}
    \end{Versuch}
    
\end{document}