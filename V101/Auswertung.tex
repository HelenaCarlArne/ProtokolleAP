\section{Auswertung} % (fold)
\label{sec:swrtng}
\subsection{Bestimmung der Winkelrichtgröße $D$}

\begin{table}
	\centering
	\caption{Messung zur Bestimmung des Eigenträgheitsmomentes der Drillachse}
	\label{tab:M1 I_D}
	\def\arraystretch{1.5} %Wie bindet man Brüche in den Text ein, sodass es hübscher aussieht und nicht so gequetscht? Hiermit :D
	\begin{tabular}{cccc}
	\toprule
	\multicolumn{2}{c}{Winkel} & {Kraft} & {Drehmoment} \\
	{$\phi$} & {$\phi/\:\text{rad}$} & {$F/\:\si{\newton}$} & {$D/\:\si{\newton\meter}$}\\
	\midrule
 45$\text{°}$ & $\frac{\pi}{4}$  & 0.22 & 0.02791\\ 
 90$\text{°}$ & $\frac{\pi}{2}$  & 0.40 & 0.02538\\
120$\text{°}$ & $\frac{2\pi}{3}$  & 0.52 & 0.02474\\
135$\text{°}$ & $\frac{3\pi}{4}$  & 0.60 & 0.02538\\
180$\text{°}$ & $\pi$              & 0.78 & 0.02474\\
225$\text{°}$ & $\frac{5\pi}{4}$  & 0.92 & 0.02335\\
240$\text{°}$ & $\frac{4\pi}{3}$  & 0.90 & 0.02141\\
270$\text{°}$ & $\frac{3\pi}{2}$  & 1.08 & 0.02284\\
315$\text{°}$ & $\frac{7\pi}{4}$  & 1.26 & 0.02284\\
360$\text{°}$ & $2\pi$              & 1.42 & 0.02252\\
	\bottomrule
	\end{tabular}
\end{table}



Mit den Messwerten aus Tabelle \ref{tab:M1 I_D} und $r=0.09965\si{\meter}$ lässt sich das Mittel der Winkelrichtgröße $D$ mit der Gleichung \eqref{eq:winkelricht} bestimmen. Die Unsicherheit ist die Standardabweichung des Mittelwertes $\sigma_D$ mit
\begin{align}
	\sigma_D &= \sqrt{\sigma_D^2} := \sqrt{\frac{1}{n^2-n} \sum_{i=1}^n{(D_i-\bar{D})^2}}
	\intertext{und}
	\bar{D} &= \frac{1}{n} \sum_{i=1}^n{D_i}.
\end{align}
Es ergibt sich $D=(0.0757\pm0.0019)\si{\newton\meter}$
\subsection{Bestimmung des Eigenträgheitsmomentes $I_D$}
Zur Bestimmung des Eigenträgheitsmomentes $I_D$ wird die verwendete Stange als nahezu masselos angenommen, 
wodurch ihr Anteil am Trägheitsmoment vernachlässigbar ist. 
Das gemessene Trägheitsmoment setzt sich aufgrund der Linearität des Trägheitsmomentes aus den Trägheitsmomenten der Massestücke $m_1$ und $m_2$ als Punktmassen, sowie dem Eigenträgheitsmoment $I_D$ zusammen,
\begin{equation}
	I= I_D+I_{m_1}+I_{m_2}.
\end{equation}
Nach Einsetzen in Gleichung \eqref{eq:Winkelricht} wird der lineare Zusammenhang von $T^2$ und $a^2$ ersichtlich.
Es gilt
\begin{align*}
	 D &= 4\mathup{\pi^{2}}\cdot\frac{I}{T^2}\\
	   &= 4\mathup{\pi^{2}}\cdot\frac{I_D+I_{m_1}+I_{m_2}}{T^2}\\
	   &= 4\mathup{\pi^{2}}\cdot\frac{I_D+a^{2}(m_1+m_2)}{T^2}\\
	T^2&= 4\mathup{\pi^{2}}\frac{I_D}{D}+4\mathup{\pi^{2}}\frac{a^{2}(m_1+m_2)}{D}\\
\end{align*}
\begin{equation}
	T^2= \underbrace{4\mathup{\pi^{2}}\frac{(m_1+m_2)}{D}}_{m_{\text{Reg}}}\cdot a^{2}+\underbrace{4\mathup{\pi^{2}}\frac{I_D}{D}}_{b_{\text{Reg}}}
\end{equation}
\begin{table}[hp]
	\centering
	\sisetup{table-format=2.3}
	\begin{tabular}{S[table-format=2.4] S[table-format=2.2] S[table-format=1.3] S[table-format=2.2] S[table-format=1.3] S[table-format=3.2] S[table-format=3.2]}
	\toprule
	{Abstand}&\multicolumn{4}{c}{Schwingungsdauer} & \multicolumn{2}{c}{Masse} \\
	{$a/\:\si{\centi\meter}$} & {$2T_{1}/\:\si{\second}$} & {$T_{1}/\:\si{\second}$} & {$2T_{2}/\:\si{\second}$} & {$T_{2}/\:\si{\second}$} & {$m_{1}/\:\si{\gram}$} & {$m_{2}/\:\si{\gram}$}\\
	\midrule
 6.4925 &  5.90 & 2.950 &  5.90 & 2.950 & 221.74 & 221.73 \\
 8.4925 &  6.55 & 3.275 &  6.46 & 3.230 & 221.74 & 221.73 \\
10.9925 &  7.29 & 3.645 &  7.21 & 3.605 & 221.75 & 221.73 \\
13.7925 &  8.66 & 4.330 &  8.63 & 4.315 & 221.76 & 221.75 \\
16.6925 & 10.07 & 5.035 & 10.10 & 5.050 & 221.75 & 221.74 \\
19.0925 & 11.13 & 5.565 & 11.15 & 5.575 & 221.75 & 221.74 \\
20.4925 & 11.92 & 5.960 & 11.76 & 5.880 & 221.75 & 221.73 \\
22.4925 & 13.03 & 6.515 & 13.00 & 6.500 & 221.76 & 221.74 \\
24.5925 & 14.10 & 7.050 & 13.96 & 6.980 & 221.75 & 221.75 \\
29.5925 & 16.67 & 8.335 & 16.73 & 8.365 & 221.76 & 221.74 \\
	\bottomrule
	\end{tabular}
	\caption{Messung zur Bestimmung des Eigenträgheitsmomentes der Drillachse}\label{tab:M2 I_D}
\end{table}
Zur Bestimmung des Eigenträgheitsmoments $I_D$ werden die gemittelten Schwingperioden-Quadrate ${T}^2$ gegen das Abstandsquadrat $a^2$ aufgetragen. Aus der Regression mittels der Formeln
\begin{subequations}
	\begin{equation}
		\Delta = N \sum{x^2} - {(\sum{x})}^2
	\end{equation}
	\begin{equation}
		m_{\text{Reg}} = \frac{N\sum{x\cdot y} - \sum{x} \cdot \sum{y}}{\Delta}
	\end{equation}
    \begin{equation}
		b_{\text{Reg}} = \frac{\sum{x^2} \cdot \sum{y} - \sum{x} \cdot \sum{x \cdot y}}{\Delta}
	\end{equation}
	\begin{equation}
		\sigma_{y} = \sqrt{\frac{\sum{(y - m_{\text{Reg}} \cdot x - b_{\text{Reg}})^2}}{N - 2}}
	\end{equation}
	\begin{equation}
		\sigma_{m} = \sigma_{y} \sqrt{\frac{N}{\Delta}}
	\end{equation}
	\begin{equation}
		\sigma_{b} = \sigma_{y} \sqrt{\frac{\sum{x^2}}{\Delta}}
	\end{equation}
\end{subequations}
% section auswertung (end)