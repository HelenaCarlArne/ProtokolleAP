\section{Auswertung} % (fold)
\label{sec:swrtng}
\subsection{Bestimmung der Winkelrichtgröße $D$}

\begin{table}
	\centering
	\caption{Messung zur Bestimmung des Eigenträgheitsmomentes der Drillachse}
	\label{tab:M1 I_D}
	\def\arraystretch{1.5} %Wie bindet man Brüche in den Text ein, sodass es hübscher aussieht und nicht so gequetscht? Hiermit :D
	\begin{tabular}{cccc}
	\toprule
	\multicolumn{2}{c}{Winkel} & {Kraft} & {Drehmoment} \\
	{$\phi$} & {$\phi/\:\text{rad}$} & {$F/\:\si{\newton}$} & {$D/\:\si{\newton\meter}$}\\
	\midrule
 45$\text{°}$ & $\frac{\pi}{4}$  & 0.22 & 0.02791\\ 
 90$\text{°}$ & $\frac{\pi}{2}$  & 0.40 & 0.02538\\
120$\text{°}$ & $\frac{2\pi}{3}$  & 0.52 & 0.02474\\
135$\text{°}$ & $\frac{3\pi}{4}$  & 0.60 & 0.02538\\
180$\text{°}$ & $\pi$              & 0.78 & 0.02474\\
225$\text{°}$ & $\frac{5\pi}{4}$  & 0.92 & 0.02335\\
240$\text{°}$ & $\frac{4\pi}{3}$  & 0.90 & 0.02141\\
270$\text{°}$ & $\frac{3\pi}{2}$  & 1.08 & 0.02284\\
315$\text{°}$ & $\frac{7\pi}{4}$  & 1.26 & 0.02284\\
360$\text{°}$ & $2\pi$              & 1.42 & 0.02252\\
	\bottomrule
	\end{tabular}
\end{table}



Mit den Messwerten aus Tabelle \ref{tab:M1 I_D} und $r=0.09965\si{\meter}$ lässt sich die Winkelrichtgröße $D$ mit der Gleichung \eqref{eq:winkelricht} bestimmen. Die Unsicherheit ist die Standardabweichung des Mittelwertes $\sigma_D$ mit
\begin{subequations}
	\begin{equation}
		\sigma_x = \sqrt{\sigma_x^2} := \sqrt{\frac{1}{n^2-n} \sum_{i=1}^n{(x_i-\bar{x})^2}}
	\end{equation}
	\begin{equation}
		\bar{x} = \frac{1}{n} \sum_{i=1}^n{x_i}.
	\end{equation}
\end{subequations}
für $x=D$.
Es ergibt sich 
\begin{equation}
	\label{wert:Winkelricht}
	D=(0.0757\pm0.0019)\si{\newton\meter}
\end{equation}
\subsection{Bestimmung des Eigenträgheitsmomentes $I_D$}
\begin{table}[hp]
	\centering
	\sisetup{table-format=2.3}
	\begin{tabular}{S[table-format=2.4] S[table-format=2.2] S[table-format=1.3] S[table-format=2.2] S[table-format=1.3] S[table-format=3.2] S[table-format=3.2]}
	\toprule
	{Abstand}&\multicolumn{4}{c}{Schwingungsdauer} & \multicolumn{2}{c}{Masse} \\
	{$a/\:\si{\centi\meter}$} & {$2T_{1}/\:\si{\second}$} & {$T_{1}/\:\si{\second}$} & {$2T_{2}/\:\si{\second}$} & {$T_{2}/\:\si{\second}$} & {$m_{1}/\:\si{\gram}$} & {$m_{2}/\:\si{\gram}$}\\
	\midrule
 6.4925 &  5.90 & 2.950 &  5.90 & 2.950 & 221.74 & 221.73 \\
 8.4925 &  6.55 & 3.275 &  6.46 & 3.230 & 221.74 & 221.73 \\
10.9925 &  7.29 & 3.645 &  7.21 & 3.605 & 221.75 & 221.73 \\
13.7925 &  8.66 & 4.330 &  8.63 & 4.315 & 221.76 & 221.75 \\
16.6925 & 10.07 & 5.035 & 10.10 & 5.050 & 221.75 & 221.74 \\
19.0925 & 11.13 & 5.565 & 11.15 & 5.575 & 221.75 & 221.74 \\
20.4925 & 11.92 & 5.960 & 11.76 & 5.880 & 221.75 & 221.73 \\
22.4925 & 13.03 & 6.515 & 13.00 & 6.500 & 221.76 & 221.74 \\
24.5925 & 14.10 & 7.050 & 13.96 & 6.980 & 221.75 & 221.75 \\
29.5925 & 16.67 & 8.335 & 16.73 & 8.365 & 221.76 & 221.74 \\
	\bottomrule
	\end{tabular}
	\caption{Messung zur Bestimmung des Eigenträgheitsmomentes der Drillachse}\label{tab:M2 I_D}
\end{table}
Zur Bestimmung des Eigenträgheitsmomentes $I_D$ wird die verwendete Stange als nahezu masselos angenommen, 
wodurch ihr Anteil am Trägheitsmoment vernachlässigbar ist. 
Das gemessene Trägheitsmoment setzt sich aufgrund der Linearität des Trägheitsmomentes aus den Trägheitsmomenten der Massestücke $m_1$ und $m_2$ als Punktmassen, sowie dem Eigenträgheitsmoment $I_D$ zusammen,
\begin{equation}
	I= I_D+I_{m_1}+I_{m_2}.
\end{equation}
Nach Einsetzen in Gleichung \eqref{eq:Winkelricht} wird der lineare Zusammenhang von $T^2$ und $a^2$ ersichtlich.
Es gilt
\begin{align*}
	 D &= 4\mathup{\pi^{2}}\cdot\frac{I}{T^2}\\
	   &= 4\mathup{\pi^{2}}\cdot\frac{I_D+I_{m_1}+I_{m_2}}{T^2}\\
	   &= 4\mathup{\pi^{2}}\cdot\frac{I_D+a^{2}(m_1+m_2)}{T^2}\\
	T^2&= 4\mathup{\pi^{2}}\frac{I_D}{D}+4\mathup{\pi^{2}}\frac{a^{2}(m_1+m_2)}{D}\\
\end{align*}
\begin{equation}
	\label{eq:Reg_ident}
	T^2= \underbrace{4\mathup{\pi^{2}}\frac{(m_1+m_2)}{D}}_{m_{\text{Reg}}}\cdot a^{2}+\underbrace{4\mathup{\pi^{2}}\frac{I_D}{D}}_{b_{\text{Reg}}}
\end{equation}

Zur Bestimmung des Eigenträgheitsmoments $I_D$ werden die gemittelten Schwingperioden-Quadrate ${T}^2$ gegen das Abstandsquadrat $a^2$ aufgetragen. Aus der Regression mittels der Formeln
\begin{subequations}
	\begin{equation}
		\Delta = N \sum{x^2} - {(\sum{x})}^2
	\end{equation}
	\begin{equation}
		m_{\text{Reg}} = \frac{N\sum{x\cdot y} - \sum{x} \cdot \sum{y}}{\Delta}
	\end{equation}
    \begin{equation}
		b_{\text{Reg}} = \frac{\sum{x^2} \cdot \sum{y} - \sum{x} \cdot \sum{x \cdot y}}{\Delta}
	\end{equation}
	\begin{equation}
		\sigma_{y} = \sqrt{\frac{\sum{(y - m_{\text{Reg}} \cdot x - b_{\text{Reg}})^2}}{N - 2}}
	\end{equation}
	\begin{equation}
		\sigma_{m} = \sigma_{y} \sqrt{\frac{N}{\Delta}}
	\end{equation}
	\begin{equation}
		\sigma_{b} = \sigma_{y} \sqrt{\frac{\sum{x^2}}{\Delta}}
	\end{equation}
\end{subequations}
für $x=a^2$, $y=T^2$ und der Identität aus Gleichung \eqref{eq:Reg:ident}, wird das Eigenträgheitsmoment $I_D$ berechnet
\begin{align}
	I_D&= \frac{D}{4\mathup{\pi^2}}b_{\text{Reg}}\\
	   &= (9.26\pm0.23)10^{-3} \si{\kilo\gram\meter\squared}
\end{align}
\begin{figure}[hp]
	\centering
	\label{fig:Regress}
	\includegraphics[width=\textwidth]{Bilder/Messung2.pdf}
	\caption{Regression von $T^2$ gegen $a^2$}
\end{figure}
\newpage
\subsection{Trägheitsmoment $I_\text{Z}$ eines Zylinders}
\label{sub:traegheitsmoment_eines_zylinders}

\begin{table}
	\centering
	
	\sisetup{table-format=2.3}
	\begin{tabular}{S[table-format=1.2] S[table-format=1.3] S[table-format=2.3] S[table-format=1.3] S[table-format=3.2]}
	\toprule
	\multicolumn{2}{c}{Schwingungsdauer} & \multicolumn{2}{c}{Abmessungen}&{Masse}\\

{$5T/\:\si{\second}$} & {$T/\:\si{\second}$} & {$\text{Höhe}\:H/\:\si{\centi\meter}$} & {$\text{Durchmesser}\:D/\:\si{\centi\meter}$} & {$M/\:\si{\gram}$}\\
	\midrule
4.41 &  0.882 &	10.120	& 9.848	& 368.57 \\
4.35 &	0.870 & 10.150	& 9.850	& 368.57 \\
4.44 &	0.888 & 10.120	& 9.844	& 368.57 \\
4.44 &	0.888 & 10.102	& 9.840	& 368.57 \\
4.36 &	0.872 & 10.112	& 9.844	& 368.57 \\
4.41 &  0.882 &	10.112	& 9.850	& 368.58 \\
4.32 &	0.864 & 10.058	& 9.850	& 368.59 \\
4.40 &	0.880 & 10.070	& 9.844	& 368.57 \\
4.33 &	0.886 & 10.068	& 9.844	& 368.58 \\
4.39 &  0.878 &	10.072	& 9.850	& 368.57 \\
	\bottomrule
	\end{tabular}
	\caption{Messung zur Bestimmung des Eigenträgheitsmomentes eines Zylinders}
	\label{tab:M3 I_Z}
\end{table}





Mit bekannter Winkelrichtgröße $D$ wird die Gleichung \eqref{eq:Winkelricht} benutzt, um das Trägheitsmoment aus der Schwingungsdauer $T$ zu berechnen. Der Fehler des Trägheitsmomentes wird mithilfe der Gausschen Fehlerfortpflanzung mit
\begin{equation}
	\label{eq:Gauss}
	\sigma=\sqrt{\sum_i^N{\left( \frac{\partial{f}}{\partial{x_i}}\right)^2\m \Delta x_i^2}}
\end{equation}bestimmt
Es ergibt sich
\begin{equation}
	\label{wert:Zylinder}
	I_\text{Z} = (1.48\pm0.04)10^{-3} \si{\kilo\gram\meter\squared}
\end{equation}
Das sichtbare Material der Zylinders ist Styropor-ähnlich. 
Da die gemessene Masse des Zylinders, $m_\text{Z} = (368.5740\pm0.0022) \si{\gram}$, stark von der theoretischen Masse eines Styropor-Zylinders gleicher Maße, $m_\text{Theorie} = \mathup{\rho_{\text{Styropor}}}V=(807.4\pm0.8) \si{\gram}$, abweicht, 
wird die Vermutung angestellt, dass der Körper ein Hohlzylinder ist oder dass der Körper nicht vollständig aus Styropor besteht.
\begin{figure}[b]
	\label{fig:tonne}
	\centering
	\includegraphics[scale=0.5]{Bilder/Tonne.pdf}
	\caption{Bezeichnungen des Hohlzylinders}
\end{figure}
Zur Bestimmung der Geometrie des vermuteten Styropor-Hohlzylinders wird die Formel für Hohlzylinder-Volumen  mit der Dichte $\mathup{\rho_{\text{Styropor}}}$ multipliziert und mit der gemessenen Masse gleichgesetzt. 
Dabei gilt die Annahme, dass die Stärke des Hohlzylinders für Mantel, Deckel und Boden gleich sind, $r_a-r_i=h_a-h_i$. 
Die so von zwei Unbekannten $r_i$ und $h_i$ auf eine Unbekannte $r_i$ reduzierte Gleichung
\begin{equation}
	m_\text{Z}=\rho(-2\pi*(r_i)^3+(r_i)^2*(2\pi r_a-\pi h_a)+(\pi h_a r_a^2)
\end{equation}
ergibt für den Innenradius $r_i = 0.0401012\si{\meter}$.

Das Trägheitsmoment eines solchen Hohlzylinders setzt sich zusammen aus den Trägheitsmomenten eines Zylindermantels und zwei Kreisscheiben, die um ihren Mittelpunkt rotieren.
\begin{align*}
	I_{\text{HZ, Theorie}}&=I_{\text{Zylindermantel}}+2I_{\text{Kreisscheibe}}\\
	I_{\text{HZ, Theorie}}&=m_\text{Mantel} \frac{r_i^2+r_a^2}{2}+m_\text{Kreisscheibe} r_i^2
\end{align*}
Man erhält
\begin{equation}
	I_{\text{HZ, Theorie}}= (6.256306729\pm Fehler)10^{-3} \si{\kilo\gram\meter\squared}
\end{equation}

\subsection{Trägheitsmoment $I_\text{K}$ einer Kugel}

\begin{table}
	\centering
	\sisetup{table-format=2.3}
	\begin{tabular}{S[table-format=1.2] S[table-format=1.3] S[table-format=1.3] S[table-format=3.1]}
	\toprule
	%Hier ist iiirgendwo ein Fehler & 
	\multicolumn{2}{c}{Schwingungsdauer} & {Abmessungen} & {Masse} \\
	{$5T/\:\si{\second}$} & {$T/\:\si{\second}$} & {$\text{Durchmesser}\:D/\:\si{\centi\meter}$} & {$M/\:\si{\gram}$}\\
	\midrule
8.60 & 1.720 & 13.745 &	812.7 \\
8.56 & 1.721 & 13.730 &	812.7 \\
8.61 & 1.722 & 13.720 &	812.7 \\
8.58 & 1.716 & 13.745 &	812.7 \\
8.56 & 1.721 & 13.750 &	812.7 \\
8.61 & 1.722 & 13.740 &	812.7 \\
8.55 & 1.710 & 13.750 &	812.7 \\
8.61 & 1.722 & 13.710 &	812.7 \\
8.60 & 1.720 & 13.745 &	812.7 \\
8.61 & 1.722 & 13.730 &	812.7 \\
	\bottomrule
	\end{tabular}
	\caption{Messung zur Bestimmung des Eigenträgheitsmomentes einer Kugel}
	\label{tab:M4 I_K}
\end{table}


Analog zu \ref{sub:traegheitsmoment_eines_zylinders} wird über die Schwingungsdauer $T$ das Trägheitsmoment $I_\text{K}$ der Kugel zu
\begin{equation}
	\label{wert:Kugel}
	I_K=(5.67\pm0.14)10^{-3} \si{\kilo\gram\meter\squared}
\end{equation}
Das sichtbare Material der Kugel ist Styropor-ähnlich. 
Für die theoretische Berechnung des Trägheitsmomentes wird eine Styropor-Vollkugel angenommen. 
Mit $I_K = \frac{2}{5} m_{\text{Kugel}} r^2$ ist 
\begin{equation}
	\label{wert:Kugel}
	I_\text{K, Theorie}= (1.5335\pm0.0010)10^{-3} \si{\kilo\gram\meter\squared}
\end{equation}


% subsection traegheitsmoment_eines_zylinders (end)
% section auswertung (end)