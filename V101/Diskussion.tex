\section{Diskussion} % (fold)
\label{sec:diskussion}

\subsection{Winkelrichtgröße $D$ und Eigenträgheitsmoment $I_\text{D}$ der Drillachse}

Mit einem relativen Fehler von ${2,51}$\% kann die Winkelrichtgröße über die statische Methode für weitere Berechnungen ausreichend genau bestimmt werden. 
Dabei liegt die größte Fehlerquelle im Ablesen der Auslenkungswinkel, welche per Augenmaß bestimmt wurden.


Das Trägheitsmoment ${I}$ setzt sich aus dem Eigenträgheitsmoment ${I_{\text{D}}}$ der Drillachse und den Trägheitmomenten $I_{\text{m}_i}$ der als punktförmig angenommenen Massestücke zusammen. 
Der verwendete Stab wird als masselos angesehen und daraus resultierend das Trägheitsmoment $I{_\text{S}}$ vernachlässigt. 
Diese Annahmen sind zu grob, sodass das errechnete Trägheitsmoment $I_\text{D}$ mit einem Fehler von $2,48$\% sehr stark vom eigentlichem Trägheitsmoment der Drillachse nach oben abweicht. 
Wird die Annahme eines masselosen Stabes nicht getroffen, so ergibt sich nach Abzug von $I_{\text{S}}$ und $I{_{\text{m}_i}}$ ein negativer Wert für $I_{\text{D}}$. 

%in Auswertung noch nichts zu geschrieben
Dies zeigt, dass der tatsächliche Wert $I_{\text{D}}$ so gering ist, dass es in weiteren Versuchsteilen vernachlässigt werden kann.

\subsection{Bestimmung der Trägheitsmomente zweier Körper}

Für die Berechnung des Trägheitsmomentes $I_{\text{Z}}$ wird die Annahme gemacht, dass es sich um einen Hohlzylinder aus Styropor handelt. 
Es ist davon auszugehen, dass der theoretische Wert nicht vollkommen korrekt ist, da der Zylinder nicht ausschließlich aus Styropor besteht. Befindet sich zur Stabilisierung ein Grundgerüst, z.B. aus Plastik oder Metall, im Zylindermantel ist bei konstant bleibender Masse ein größerer Teil dieser noch weiter von der Drehachse entfernt. Dies liefert ein größeres Trägheitsmoment und entspräche dann eher dem gemessenen Wert und würde die Abweichung von $81,84$\% erklären.
%%%%hier fehlt das richtige Trägheitsmoment des Hohlzylinders, welches gestern berechnet wurde%%%%
Das gemessene Trägheitsmoment hat einen relativen Fehler von $2,7\%$, kann also ziemlich genau gemessen werden.

Das gemessene Trägheitsmoment $I{_\text{K}}$ der Kugel mit einem relativen Fehler von ${2,45}$\% liegt in der Größenordnung des errechneten Wertes, der eine Abweichung von ${0,65}$\% aufweist. Dabei weicht dieser um ${27,05}$\% von der Messung ab.

\subsection{Bestimmung der Trägheitsmomente der Modellpuppe}
\subsubsection{Gemessene Trägheitsmomente}

Das über die gemessene Periodendauer berechnete Gesamtträgheitsmoment der ersten Position ist $36,52$\% größer als das der Zweiten, da in Position 1 sich mehr Masse weiter von der Drehachse entfernt befindet.
Die relativen Fehler von $2,21$\% (Position 1) und $2,44$\% (Position 2) liegen in einem vertretbaren Rahmen.

Die berechneten Trägheitsmomente von Kopf und Rumpf der Puppe weisen einen relativen Fehler von $5,66$\% bzw. $11,67$\% auf. 
Die Trägheitsmomente der Arme unterscheiden sich in Position 1 um $21,05$\%, in Position 2 um $30,23$\%. 
Die Trägheitsmomente der Beine mit relativen Fehlern von $10,33$\% (links) und $12,50$\% unterscheiden sich um $6,67$\%. 
Dies ist erklärbar mit den nicht genau bestimmbaren Abmessungen der Puppe aufgrund der unregelmäßigen Form.

Wegen der ausgesprochen groben Näherung des Puppenkörpers durch einzelne Zylinder und Kugeln lassen sich nur die Verhältnisse der Momente, nicht aber diese selbst, vergleichen.
 Aus den Quotienten der theoretischen Werte $Q_\text{T}=1,73 \pm 0,08$ und dem der gemessenen Werte $Q_\text{M}=1,58 \pm 0,00$ folgt eine  Übereinstimmung von $109,96 \pm 5,13$\%. 
Die Abweichung resultiert aus der Tatsache, dass die Position von Armen und Beinen sich während der Schwingung verändert, sodass sie sich am Ende der Messung nicht mehr senkrecht zur Drehachse befinden, sondern in einem unbestimmten Winkel geneigt sind.

%Ende:
Größte Fehlerquelle in allen Versuchsteilen ist die manuelle und einmalige Zeitmessung. Bei der Fehlerrechnung wird die menschliche Reaktionszeit nirgends berücksichtigt. Der Fehler kann etwas minimiert werden, wenn statt nur einer Schwingungsperiode die Zeit für mehrere Schwingungen gemessen und anschließend durch deren Anzahl dividiert wird.
Zusätzlich ist, besonders bei schnellen Schwingungen, ein genaues Erkennen des Endes einer Periode schwierig. Der Versuch eignet sich deswegen besonders für die Messung großer Trägheitsmomente. Die Näherungen, sowohl bei den Körpern, als auch der Modellpuppe, sind sehr grob. Die Annahme einer homogenen Masseverteilung passt nicht genau mit der Realität zusammen.
Betrachtet man die Fehlerquellen stimmen theoretische und gemessene Werte soweit überein, dass von einer Richtigkeit der Formeln ausgegangen und der Satz von Steiner als verifiziert betrachtet werden kann.

\end{document}
% section diskussion (end)
