
\section{Ziel}
\label{sec:ziel}
Es werden die Trägheitsmomente verschiedener Körper gemessen und anschließend mit den theoretisch errechneten Werten verglichen. Hierzu werden die Winkelrichtgröße $D$ und das Trägheitsmoment der Drillachse $I_{\mathup{D}}$ bestimmt.

\section{Theorie} 
\label{sec:theorie}
Translation und Rotation verbinden Analogien.
Bei Drehbewegungen sind das Drehmoment $\vec{M}$, das Trägheitsmoment $I$ und die Winkelbeschleunigung $\vec{\omega}$ maßgebliche Größen. Das Drehmoment $\vec{M}$ mit
\begin{equation}
	\label{eq:Moment_Kraft}
	\vec{M}=\vec{r}\times \vec{F}\\
\end{equation}
ist abhängig von der Kraft $\vec{F}$, welche  im Abstand $|\vec{r}|$ von der Drehachse angreift. 
Ausgedrückt über die Winkelrichtgröße $D$ und über die Auslenkung des Winkels $\phi$ ist der Betrag des Drehmomentes 
\begin{equation}
	\label{eq:Moment_Winkelricht}
	|\vec{M}|=D\phi.
\end{equation}
Das Trägheitsmoment $I$ ist, analog zur Masse $m$ in Translation, der Widerstand eines Drehmoments $\vec{M}$.
Es gilt für Drehachsen durch den Masseschwerpunkt $S$
\begin{align}
	\label{eq:Tragheit}
	I_{\mathup{S}} &= \sum_{i=1}^n m_{i}\cdot r_{i}^{2}
		\intertext{für diskrete Massestücke $m_{i}$ mit dem Abstand $r_i$ von der Rotationsachse und}
	I_{\mathup{S}} &=\int_{m} {r_{i}^{2}}\mathup{d}m\\ 
	  &=\int_{V_{\text{Körper}}} \rho{\vec{r}}{r_{i}^{2}}\mathup{d}V
\end{align}
für kontinuierliche Masseverteilungen mit Massenverteilung $\rho{\vec{r}}$.
Ist die Drehachse um $\mathup{a}$ parallel zur Achse durch den Schwerpunkt verschoben, so kann das Trägheitsmoment mit dem Satz von Steiner, 
\begin{equation}
	\label{eq:steiner}
	I_{\mathup{a}}= I_{\mathup{S}}+m\cdot\mathup{a^2}, 
\end{equation}
berechnet werden. 
%Das Trägheitsmoment $I_{\mathup{a}}$ setzt sich zusammen aus dem Trägheitsmoment des Körperschwerpunkts $I_{\mathup{S}}$ und dem Produkt aus der Gesamtmasse $m$ des Körpers und dem Quadrat des senkrechten Abstands der eigentlichen Drehachse die um $a$ parallel zur Schwerpunktsachse verschoben ist.


Mechanische Drehschwingungen führen harmonische Schwingungen mit der Schwingungsdauer
\begin{equation}
	\label{eq:Schwingperiode}
	T=2\mathup{\pi} \sqrt{\frac{I}{D}}
\end{equation}
für kleine Auslenkungswinkel $\phi$ aus. 
Die Winkelrichtgröße $D$ berechnet sich mit Formel \eqref{eq:Moment_Kraft} und \eqref{eq:Schwingperiode} zu
\begin{equation}
	\label{eq:Winkelricht}
	D= 4\mathup{\pi^{2}}\cdot\frac{I}{T} =\frac{F\cdot r}{\phi},
\end{equation}


wobei die Auslenkung senkrecht %... %[ #fehlt: keine Formulieridee].

% section theorie (end)
