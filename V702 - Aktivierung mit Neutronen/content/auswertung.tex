\newpage
\section{Auswertung}
\label{sec:Auswertung}
Alle in der Auswertung benutzten Mittelwerte werden mithilfe der Gleichung
\begin{equation}
\tilde{x}=\frac{1}{n}\sum_{i=1}^n {x_i}
\end{equation}
bestimmt. Die Standardabweichungen der Mittelwerte ergeben sich mit der Formel
\begin{equation}
\mathup\Delta{\tilde{x}}=\sqrt{\frac{1}{n(n-1)}\sum_{i=1}^n {(x_i-\tilde{x})²}}.
\end{equation}
Wird eine Größe beziffert, welche sich aus fehlerbehafteten Daten zusammensetzt, wird der absolute Fehler über die Gaußsche Fehlerfortpflanzung erhalten.
Hierzu gilt
\begin{equation}
\mathup\Delta{f}(x_1,..,x_n)=\sqrt{\left(\frac{\mathup{d}f}{\mathup{d}x_1}\Delta{x_1}\right)²+..+\left(\frac{\mathup{d}f}{\mathup{d}x_n}\Delta{x_n}\right)²}.
\end{equation}
Zur Berechnung aller Größen werden rechnerintern nicht-gerundete Größen mit maximaler Anzahl der Nachkommastellen benützt.
Die angegeben Werte werden auf die erste signifikante Stelle des Fehlers gerundet.

\subsection{Bestimmung der Hintergrundstrahlung und statistische Betrachtung}
\label{sec:Auswertung_hintergrund}
Die statistische Verteilung der Zerfälle sowie die natürlich auftretende Strahlung nehmen Einfluss auf die Zuverlässigkeit der Messwerte.
Um genaue statistische Aussage treffen zu können, werden die im Zeitintervall $\mathup\Delta t$ detektierten Zerfälle $\mathup\Delta N$ gemäß
\begin{equation}
	n=\frac{\mathup\Delta N}{\mathup\Delta t}
\end{equation}
bestimmt.
Die Fehler werden anders als im Auswertungspräambel nicht durch wiederholte Messung bestimmt, sondern mit
\begin{equation}
	\mathup\Delta n=\frac{\sqrt{\mathup\Delta N}}{\mathup\Delta t}
\end{equation}
als gegeben angenommen.
Von den Messungen wird die im Voraus gemessene Hintergrundstrahlung abgezogen. 
Für diese gilt im Folgenden
\begin{equation}
	n_0=\frac{166}{\SI{900}{\second}}\approx\SI{0.18(1)}{\hertz},
	%=\frac{\mathup\Delta N}{\mathup\Delta t}=
\end{equation}
das Herausrechnen der Hintergrundstrahlung bei Messwert $N_\text{Messung}$, welche im Zeitintervall $\mathup\Delta t$ aufgenommen wurde, folgt der Formel
\begin{equation}
	N=N_{\text{Messung}}-n_0\cdot\mathup\Delta t.
\end{equation}


%\subsection{Zerfallsgesetz}
%Der radioaktiver Zerfall wird durch exponentiellen Abfall beschrieben.
%Hierfür gilt in guter Näherung
%\begin{equation}
%	N(t)=N_0 \exp(-\lambda t),
%	\label{eq:Zerfallsgesetz}
%\end{equation}
%welches beim Auftrag mit halblogarithmischer Skalierung eine lineare Regression,
%\begin{equation}
%	\ln(N(t))=\ln(N_0(\exp(-\lambda t))=\ln(N_0)+\ln(\exp(-\lambda t)) =\underbrace{\ln(N_0)}_{A_\text{Reg}}-\underbrace{\lambda}_{m_\text{Reg}} \cdot t,
%	\label{eq:Zerfallsgesetz_linear}
%\end{equation}
%zulässt.
%Charakteristische Größe bei der Betrachtung von Zerfällen ist die Halbwertzeit \ref{sec:Theorie}, Gleichung \eqref{eq:halbwertszeit}, die mit der Zerfallskonstante $\lambda$ in Verbindung steht.

\subsection{Messung von Indium}
\label{sec:Auswertung_indium}
Bei der Untersuchung von Indium wird in $\SI{220}{\second}$-Intervallen die Anzahl der detektierten Zerfälle aufgenommen. 
Nach der Messwertaufnahme am Ende eines Zeitintervalls wird der Zähler auf Null zurückgesetzt und solange eine weitere Messung gestartet, bis kummuliert eine Stunde gemessen wurde.
In Tabelle \ref{tab:indium} sind die Messwerte $N_\text{Messung}$, deren korrigierte Werte $N$ und der Messzeitpunkt $t$ abgebildet.
\begin{table}[htp]
	\centering
		\begin{tabular}{S[table-format=4.0]
                        S[table-format=4.0]
                        S}
			\toprule
			{$t\,\text{in}\,\si{\second}$} & {$N_\text{Messung}$} & {$N$}\\
			\midrule
			 250 & 2038 &  2000(40)\\
			 500 & 1994 &  1950(40)\\
			 750 & 1867 &  1820(40)\\
			1000 & 1740 &  1700(40)\\
			1250 & 1611 &  1560(40)\\
			1500 & 1608 &  1560(40)\\
			1750 & 1473 &  1430(40)\\
			2000 & 1415 &  1370(30)\\
			2250 & 1311 &  1260(30)\\
			2500 & 1234 &  1190(30)\\
			2750 & 1213 &  1170(30)\\
			3000 & 1200 &  1150(30)\\
			3250 & 1055 &  1010(30)\\
			3500 &  971 &   920(30)\\
			3750 &  901 &   850(30)\\
			\bottomrule
		\end{tabular}
	\caption{Messwerte: Zerfälle bei der Messung von Indium beim Zeitintervall von \SI{250}{\second}.}
	\label{tab:indium}
\end{table}
\begin{figure}[p]
    \centering
    \includegraphics[width=0.8\textwidth]{Bilder/indium.pdf}
    \caption{Korrigierte und logarithmierte Anzahl der Zerfälle für Indium aufgetragen gegen die Zeit.}
    \label{fig:indium}
\end{figure}
\begin{figure}[p]
    \centering
    \includegraphics[width=0.8\textwidth]{Bilder/rhodium.pdf}
    \caption{Korrigierte und logarithmierte Anzahl der Zerfälle für Rhodium aufgetragen gegen die Zeit.}
    \label{fig:rhodium}
\end{figure}

In Abbildung \ref{fig:indium} sind die Zerfälle $N$ pro Zeitintervall gegen die fortschreitende Zeit aufgetragen, um Aussage über den Zerfallsprozess treffen zu können.
Die Messwerte sind gemäß der Überlegung im  Abschnitt \ref{sec:zerfall} logarithmiert, die Skaleneinteilung verbleibt linear.
Die Regression ergibt 
\begin{align}
	m &=\SI{-0.000232(7)}{}\\
	b &=\SI{7.67(1)}{}\\
	\intertext{und mit den Gleichungen \eqref{eq:halbwertszeit} und \eqref{eq:startwert}}\\
	T_{\sfrac{1}{2}}&= \SI{2980(90)}{\second}\\
	N_0&=\SI{38000(1000)}{},
\end{align}

\subsection{Messung von Rhodium}
Analog zu Abschnitt \ref{sec:Auswertung_indium} wird die Messung von Rhodium durchgeführt.
Das Zeitintervall $\mathup\Delta t$ beträgt $\SI{20}{\second}$,
in Tabelle \ref{tab:rhodium} werden die Messwerte $N_\text{Messung}$, deren korrigierte Werte $N$ und der Messzeitpunkt $t$ gezeigt.
\begin{table}[htp]
	\centering
		\begin{tabular}{S[table-format=2.0]
                        S[table-format=3.0]
                        S}
			\toprule
			{$t\,\text{in}\,\si{\second}$} & {$N_\text{Messung}$} & {$N$}\\
			\midrule
				20	&	545	&540(20)\\
				40	&	392	&390(20)\\
				60	&	329	&330(20)\\
				80	&	238	&230(20)\\
				100	&	221	&220(20)\\
				120	&	178	&170(10)\\
				140	&	122	&120(10)\\
				160	&	112	&110(10)\\
				180	&	 90	&86(9)\\
				200	&	 72	&68(8)\\
				220	&	 54	&50(7)\\
				240	&	 71	&67(8)\\
				260	&	 40	&36(6)\\
				280	&	 51	&47(7)\\
				300	&	 35	&31(6)\\
				320	&	 34	&30(6)\\
				340	&	 38	&34(6)\\
				360	&	 28	&24(5)\\
				380	&	 34	&30(6)\\
				400	&	 22	&18(4)\\
				420	&	 26	&22(5)\\
				440	&	 20	&16(4)\\
				460	&	 19	&15(4)\\
				480	&	 22	&18(4)\\
				500	&	 14	&10(3)\\
				520	&	 19	&15(4)\\
				540	&	 21	&17(4)\\
				560	&	 15	&11(4)\\
				580	&	 20	&16(4)\\
				600	&	 23	&19(5)\\
				620	&	 18	&14(3)\\
				640	&	 24	&20(5)\\
				660	&	 11	&7(3)\\
				680	&	 13	&9(3)\\
				700	&	 19	&15(4)\\
				720	&	 11	&7(3)\\
				740	&	 15	&11(4)\\
				760	&	 14	&10(3)\\
				780	&	 18	&14(4)\\
				800	&	 13	&9(3)\\
			\bottomrule
		\end{tabular}
	\caption{Messwerte: Zerfälle bei der Messung von Rhodium beim Zeitintervall von \SI{20}{\second}.}
	\label{tab:rhodium}
\end{table}
In Abbildung \ref{fig:rhodium} sind die Zerfälle $N$ pro Zeit gegen die fortschreitende Zeit aufgetragen.

Anders als Indium wird bei Rhodium die Überlagerung zweier Zerfälle angenommen.
Mit der weiteren Annahme, dass die beiden Zerfälle unterschiedliche Halbwertszeiten haben und 
dass das kurzlebige Isotop deutlich schneller zerfällt als das langlebige Isotop, werden die Messwerte in zwei Gruppen aufgespalten.
In Abbildung \ref{fig:rhodium} sind die Messwerte aufgetragen und mit Farben ihrer Gruppe zugeordnet.
Es wird die zweite Gruppe mit dem langlebigen Rhodium identifiziert.
Diese Gruppe wird durch Vergleich der momentanen Steigung im Graph gefunden, die Elemente teilen eine gemeinsame Steigung.
Die lineare Regression dieser Werte ergibt 
\begin{align}
	m &=\SI{-0.0018(4)}{}\\
	b &=\SI{3.8(2)}{}\\
	\intertext{und mit den Gleichungen \eqref{eq:halbwertszeit} und \eqref{eq:startwert}}\\
	T_{\sfrac{1}{2}}&= \SI{380(80)}{\second}\\
	N_0&=\SI{120(30)}{}.
	\label{qu:rhodium1}
\end{align}

Mit dem bekannten Zerfallverhalten aus der vorangegangen Regression kann das Zerfallverhalten des kurzlebigen Rhodiums bestimmt werden.
Die Regressionsparameter $m,b$ sind linear, der erste Teil des Graphen wird als Überlagerung beider Zerfälle angesehen.
Die Regression des ersten Teiles unter Berücksichtigung der Regressionsparameter von \eqref{qu:rhodium1} ergibt
\begin{align}
	m &=\SI{-0.0100(4)}{}\\
	b &=\SI{2.6(2)}{}\\
	\intertext{also}\\
	\lambda &=\SI{0.0118(6)}{}\\
	T_{\sfrac{1}{2}}&= \SI{59(3)}{\second}\\
	N_0&=\SI{2.8(6)e04}{}.
\end{align}
