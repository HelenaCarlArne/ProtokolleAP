\section{Durchführung}
\label{sec:Durchfuehrung}
\subsection{Aufbau der Messapparatur}
\begin{figure}
\begin{minipage}[l]{0.49\textwidth}
	\centering
	\includegraphics[width=0.4\textwidth]{Bilder/Aufbau1.pdf}
	\caption{Aufbau der Messapparatur. \cite{V102}}
	\label{fig:aufbau1}
\end{minipage}
\begin{minipage}[r]{0.49\textwidth}
	Die Periodendauer der Torsionsschwingung wird über eine elektronische Stoppuhr gemessen. 
	Mit Schwingungsbeginn soll das Zählwerk starten und nach einer Periode enden. Die Periodendauer $T$ kann dann direkt am Zählwerk abgelesen werden.
	Dieser Vorgang wird realisiert über eine, durch die in Abbildung \ref{fig:aufbau2} gezeigte Lichtschranke mit steuerbare Torstufe. 
	Das Licht einer Lampe wird zunächst durch eine Sammellinse gebündelt und anschließend durch einen Spalt auf den Spiegel am Torsionsdraht geworfen. Wird der Draht durch Bewegen des Justierrades zu Schwingungen angeregt, passiert der reflektierte Lichtstrahl dabei die Photodiode. 
	Sobald der Lichtstrahl auf die Photodiode trifft, erzeugt diese ein elektrisches Signal, welches durch eine geeignete Schaltung auf die Torsteuerungseingänge des Zählwerks geleitet wird.
\end{minipage}
\end{figure}

Abbildung \ref{fig:aufbau1} zeigt die Messapparatur. 
Am unteren Ende eines einseitig fest eingespannten Drahtes ist eine Kugel befestigt, in deren Innern sich ein Permanentmagnet befindet. 
Sie hängt zwischen \textsc{Helmholtz}-Spulen, die ein annähernd homogenes Magnetfeld erzeugen.
In der unteren Drahthälfte wird dieser durch einen kleinen Spiegel unterbrochen, der zur Bestimmung der Periodendauer benötigt wird.
\label{sec:durchfuehrung2}
\begin{figure}[hbp]
	\centering
	\includegraphics[width=0.5\textwidth]{Bilder/Aufbau2.pdf}
	\caption{Aufbau der Lichtschranke. \cite{V102}}
	\label{fig:aufbau2}
\end{figure}
% Die Periodendauer der Torsionsschwingung wird über eine elektronische Stoppuhr gemessen. 
% Mit Schwingungsbeginn soll das Zählwerk starten; nach einer Periode enden. Die Periodendauer $T$ kann dann direkt am Zählwerk abgelesen werden.
% Dieser Vorgang wird realisiert über eine, durch die in Abbildung \ref{fig:aufbau2} gezeigte Lichtschranke ,steuerbare Torstufe. 
% Das Licht einer Lampe wird zunächst durch eine Sammellinse gebündelt und anschließend durch einen Spalt auf den Spiegel am Torsionsdraht geworfen. Wird der Draht durch bewegen des Justierrades zu Schwingungen angeregt wandert der Lichtstrahl von einer Seite zur anderen und passiert dabei die Photodiode. 
% Sobald der Lichtstrahl auf die Photodiode trifft, erzeugt diese ein elektrisches Signal, welches durch eine geeignete Schaltung auf die Torsteuerungseingäng des Zählwerks geleitet wird.
\subsection{Bestimmung der gesuchten Größen}
Zur Berechnung der Module müssen Drahtlänge und -durchmesser bekannt sein. 
Gemessen werden diese mit einer Mikrometerschraube und einem Maßband.
Übrige Systemparameter sind bekannt.

In drei Versuchsteilen wird der Draht durch Auslenken des Justierrades zu harmonischen Schwingungen angeregt und die am Zählwerk angezeigte Schwingungsdauer notiert.
Zur Bestimmung des Schubmoduls $G$ muss der Permanentmagnet $m$ parallel zum Draht und senkrecht zum Erdmagnetfeld ausgerichtet sein, um nicht beeinflusst zu werden
Es werden zehn Messwerte aufgenommen.

Anschließend wird das magnetische Moment $m$ des Permanentmagneten bestimmt. 
Hierzu werden die \textsc{Helmholtz}-Spulen eingeschaltet und der Permanentmagnet $m$ parallel zum Magnetfeld der Spulen ausgerichtet, um den maximalen Einfluss des Magnetfeldes auf die Schwingung zu erreichen.  
Die Spulenstrom $I$ wird von $I=\SI{0.1}{\ampere}$ bis $I=\SI{1}{\ampere}$ in fünf Schritten variiert und jeweils fünf Schwingungsdauern notiert.\\
Zur Bestimmung des Erdmagnetfelds wird die Messung mit abgeschalteten \textsc{Helmholtz}-Spulen wiederholt.
Es werden zehnmal eine Periode gemessen.
