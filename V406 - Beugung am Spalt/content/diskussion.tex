\section{Diskussion}
\label{sec:Diskussion}
Die mittels beider Methoden bestimmten Abmessungen $b_i$ und $g_3$ weisen geringe Abweichungen mit \\$\mathup{\Delta}{b_1}=0,92\,\%$,$\mathup{\Delta}{b_2}=0,54\,\%$, $\mathup{\Delta}{b_3}=7,1\,\%$ und $\mathup{\Delta}{g_3}=1,22\,\%$ voneinander auf. 
Die Abweichungen $\tilde{\mathup{\Delta}}{b_i}$ der Spaltbreiten im Bezug zu den Herstellerangaben sind in Tabelle \ref{tab:rel_f} aufgetragen. 

\begin{table}
	\centering
	\sisetup{table-format=2.2}
	\begin{tabular}{S S S}
	\toprule
\multicolumn{1}{c}{Blende} & \multicolumn{2}{c}{relativer Fehler /\:\%} \\
{$\,$} &{$\text{Beugung}$} & {$\text{Mikroskop}$} \\	
	\midrule
$\tilde{\mathup{\Delta}}{b_1}$ &  1,47 &26,67\\
$\tilde{\mathup{\Delta}}{b_2}$ &  7,25 &7,00\\
$\tilde{\mathup{\Delta}}{b_3}$ &  1,00 &43,00\\
$\tilde{\mathup{\Delta}}{g_3}$ & 22,50 &31,00\\
\bottomrule
\end{tabular}
\caption{Relative Fehler der Spaltbreiten bei verschiedenen Messmethoden.}
\label{tab:rel_f}
\end{table}

Deutlich zu erkennen ist, dass die mit Hilfe des Mikroskops ermittelten Spaltbreiten wesentlich stärker von den Herstellerangaben abweichen als die über die Beugung gemessenen Werte.
Grund dafür ist die ungenaue Bestimmung der Vergrößerung über Vergleichsskalen.
Außerdem spielen Ablesefehler eine größere Rolle als bei der Beugung.

Bei der Messmethode via Diffraktion kann durch die erhebliche Anzahl an Messwerten der Ablesefehler relativ gering gehalten werden.
 Bei der Vermessung des breiten Einzelspaltes besitzt der relative Fehler einen höheren Wert als die anderen Blenden. 
Dies lässt sich auf das besonders ausgeprägte Maximum und unscheinbare Nebenmaxima zurückführen. 
Durch die umfangreichere Aufnahme von Messwerten, z.B. durch Reduzierung der Schrittweite auf $\SI{0,5}{\milli\meter}$ im Bereich des Hauptmaximums, ließe sich diese Unsicherheit eventuell verringern.
Der Vorgang der Messung ist bei genauer Justierung der Anordnung wenig fehleranfällig.
Bedeutenste Fehlerquelle ist die Beeinflussung der Intensitätsmessung durch sich ändernde Lichtverhältnisse. 
Es gilt daher die Beleuchtungsverhältnisse so weit wie möglich konstant zu halten.

