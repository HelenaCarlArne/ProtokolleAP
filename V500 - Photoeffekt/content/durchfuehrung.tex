\section{Durchf\"uhrung}
\label{sec:Durchfuehrung}
Mithilfe eines Prismas wird das Licht einer Halogenlampe spektral zerlegt.
Dies wird als Lichtquelle mit seriell monochromatischen Licht einstellbarer Wellenlänge $\lambda$ betrachtet.
Über eine Vorrichtung und das Tarieren der Skala kann die Wellenlänge $\lambda$ festgestellt werden.
\subsection{Gegenspannungsmethode}
Nach der Theorie \ref{sec:theorie1} ist laut der \textsc{Einstein}schen Gleichung \eqref{eq:einstein} eine Restenergie in Form einer Spannung $W = q\cdot U$ zu erwarten, die mithilfe einer Gegenspannung $U_0$ kompensiert werden kann. 
Für diesen Fall geht der Photostrom $I$ gegen Null.
Der Aufbau ist in Abbildung \ref{fig:aufbau}
Für die verfügbaren Spektrallinien der Halogenlampe werden die Wellenlängen $\lambda_i$ bestimmt. 
Anschließend wird über die Spannungsquelle $U$ eine Gegenspannung $U_\text{G}$ eingestellt, sodass das Amperemeter den Photostrom $I=0$ anzeigt.

Das Auswerten der Daten erfolgt über das Auftragen der Gegenspannung $U_0$ gegen die verwendete Wellenlänge $\lambda$.
Die lineare Regression der \textsc{Einstein}schen Gleichung \eqref{eq:einstein} ergibt
\begin{equation}
	Bla.
\end{equation}
