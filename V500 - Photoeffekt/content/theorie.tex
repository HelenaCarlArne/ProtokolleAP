\section*{Ziel}
Es soll der Photo-Effekt mithilfe eines von dem Originalaufbau abgewandelten Aufbau gezeigt sowie den Wert des Planckschen Wirkungsquantum $\hslash$ berechnet werden, indem mit der Gegenspannungsmethode die Nullspannung gegen die Lichtfrequenz $f$ aufgetragen wird.
\section{Theorie}
\label{sec:Theorie}
Der Photo-Effekt wurde 1896 von Wilfried \textsc{Hallwachs} entdeckt.
Der Originalaufbau ist in Abbildung \ref{fig:originalaufbau} gezeigt.
Trifft auf ein elektrisch negativ geladenen Material Licht einer gewissen Wellenlänge $\lambda$ respektive Lichtfrequenz $f$, so entlädt sich das Material. 
Dies wird für positive Ladungen nicht beobachtet, auch nicht bei Licht der Wellenlänge $\lambda$ kleiner als eine materialspezifische Konstante.

