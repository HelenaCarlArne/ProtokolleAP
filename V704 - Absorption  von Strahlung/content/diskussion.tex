\newpage
\section{Diskussion}
\label{sec:Diskussion}
\subsection{Bestimmung von \texorpdfstring{$\mu$}{Absorptionskoeffizient} und \texorpdfstring{$N_0$}{Anfangswert} für Blei und Eisen mit Hilfe von \texorpdfstring{$\gamma$}{Gamma}-Strahlung}
Mit $0,84\,\%$ hat der Absorptionskoeffizient von Blei einen sehr geringen relativen Fehler. Auch der Fehler von Eisen mit $10,00\,\%$ liegt noch in einem annehmbaren Rahmen. Es zeigt sich, dass das Verfahren zur Bestimmung der Absorptionskoeffizienten gut geeignet und die Messwerte leicht reproduzierbar sind.
\subsection{Maximale Energie der \texorpdfstring{$\beta^-$}{Beta}-Strahlung}
Die Reichweite der Elektronen hat einen relativen Fehler von $6,89\,\%$; der Fehler der Energie ist $23,33\,\%$. Vom Literaturwert $E_\mathup{lit}=0,294\,\si{\mega}\mathup{e}\si\volt$ weicht der experimentell bestimmte Wert um $2,04\,\%$ ab. Die Methode erweist sich als sehr genau; größte Fehlerquelle ist der Nulleffekt. Dieser kann nie komplett ausgeschlossen werden, jedoch aber durch eine genaue Messung für ein genügen großes $t$ bestimmt und bei der Auswertung berücksichtigt werden.
