\section{Ziel}
\label{sec:Ziel}
Es wird die Wechselwirkung energiereicher Strahlung mit Materie untersucht.
Hierzu wird die Reichweite und die Transmission von $\beta$- und $\gamma$-Strahlung durch Blenden betrachtet sowie der Wirkungsquerschnitt $\sigma$ und Absorptionskoeffizient $\mu$ verschiedener Materialien ermittelt.

\section{Theorie}
\label{sec:Theorie}
Die $\beta$-Stahlung ist Materie-Strahlung von (schnellen) Elektronen, $\gamma$-Strahlung ist Photonen-Strahlung.
% im Bereich von \SI{60}{\kilo}eV bis \SI{1300}{\kilo}eV.
\subsection{Betrachtung der $\gamma$-Strahlung, Absorptionsgesetz}
\label{sec:gamma}
Für die $\gamma$-Strahlung wird ein Absorptionsgesetz beschrieben.
Beim Eindringen in die Materie treten die Photonen mit den Atomen in Wechselwirkung; 
dabei nimmt die Anzahl der Teilchen pro Zeit und Fläche ab. 
Als Maß für die Häufigkeit der Wechselwirkung kann der Wirkungsquerschnitt $\sigma$ genutzt werden.
$\sigma$ ist dabei die Fläche bestimmter Größe, die jedem Teil des Absorbers zugeordnet wird. 
Es wird die Wahrscheinlichkeit $W$ beschrieben, mit welcher ein eintreffendes Photon mit dem Atom wechselwirkt. 
Dabei gilt die Beziehung 
\begin{equation}
W=n D \sigma =\frac{n D \sigma F}{F}.
\end{equation}
Dabei ist $D$ die Dicke des Absorbers und $F$ die Querschnittsfläche. 
$n$ beschreibt die Anzahl der Teilchen pro Volumeneinheit.
Treffen $N_0$ Teilchen in einem Zeitintervall auf die Fläche $F$ kann über
\begin{equation}
N=N_0 n D \sigma
\end{equation}
die Anzahl $N$ der Wechselwirkungen im gewählten Zeitintervall bestimmt werden. 
In einem realen Absorber überdecken sich die Volumeneinheiten der Anzahl $n$ teilweise. Dies hat zur Folge, dass die Überdeckung nur vernachlässigbar ist, wenn eine dünne Schicht $dx$ betrachtet wird in der $dN$ Reaktionen stattfinden. Damit ist
\begin{equation}
dN=-N(x) dx n \sigma.
\end{equation}
Die Menge der Teilchen, die erst nach der Strecke $dx$ mit der Materie wechselwirken, nimmt um $N(x)$ ab. Wird GLeichung y integriert, also die gesamte Dicke $D$ betrachtet ergibt sich das Absorptionsgesetz

\begin{equation}
N(D)=N_0 exp(-n \sigma D).
\end{equation}
\subsection{Betrachtung der $\beta$-Strahlung, Reichweite}
\label{sec:beta}
Anders als bei $\gamma$-Strahlung wird für  $\beta$-Strahlung kein geschlossenes Absorptionsgesetz in vergleichbarer Kürze beschrieben.
Die $\beta$-Strahlung tritt über verschiedene Prozesse mit Materie in Wechselwirkung.
