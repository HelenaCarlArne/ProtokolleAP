\section{Diskussion}
\label{sec:Diskussion}
Die gemessenen Werte der Abschirmungszahl für Natrium, Kalium und Rubidium weisen sehr geringe absolute Fehler auf. Die Abschirmungszahlen weichen um $12,9\%$, $0,6\%$ und $0,3\%$ von den Literaturwerten ab. Diese sind gegeben als $7,46$, $13,06$ und $26,95$.\cite{lit} Auffällig ist, dass der relative Fehler von Natrium gegenüber Kalium und Rubidium sehr groß ist. Dies liegt an den sehr nahe beieinanderliegenden Dublettlinien, welche eher schwierig zu messen sind. Da das lange Schauen durch das Fernrohr für die Augen sehr ermüdend ist, könnte ein längerer Messzeitraum die Fehler verringern. Es wurde nicht überprüft, ob der Reflexionswinkel exakt $2\beta$ beträgt. Dies könnte eine weitere Fehlerquelle sein, da $\beta$ durchaus für die Auswertung relevant ist.
Für eine genauere Eichung des Okularmikrometers wäre es sinnvoll, mehrere Messungen durchzuführen und den fehlerbehafteten Eichfaktor mit in die Berechnung der Abschirmungszahlen einfließen zu lassen. Die Bestimmung der Abschirmungszahlen mit Hilfe eines Gitterspektralapparates ist zu empfehlen.
