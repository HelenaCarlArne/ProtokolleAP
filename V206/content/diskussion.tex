\section{Diskussion}
\label{sec:Diskussion}
\subsection{Effizienz der Wärmepumpe}
Die Güteziffer der verwandten Wärmepumpe beträgt ##Wert_güteziffer##.
Die geringe Unsicherheit in der Güteziffer ist Maß dafür, dass der errechnete Mittelwert den realen Wert repräsentiert.

Die Effizienz der Wärmepumpe ist hoch, aus ökologischer und wirtschaftlicher Sicht ist die Verwendung einer Wärmepumpe in vergleichbarer Situation wie im Versuch empfehlenswert.

\subsection(Vergleich der verwandten Wärmepumpe mit großtechnischen Anlagen) 
Der Vergleich dieser Güteziffer mit den Werkangaben von industriell genutzten Wärmepumpen, etwa in Klimaanlagen oder in großtechnische Kühlgeräten, zeigt, dass die verwandte Wärmepumpe ähnlich effizient arbeitet.

\subsection{Fehler durch Messung}
Der Versuch zeichnet sich dadaurch aus, dass er weitestgehend störunanfällig ist.
Die Unsicherheiten in Tmeperatur, Druck und Leistung ist unbekannt, eine Fehlerdiskussion wird dadurch erschwert.
Mögliche Fehlerquellen bestehen darin, dass die Temperatur der Reservoire zu Beginn des Experimentes voneinander abweichen könnten.
Die Startwerte der Temperaturen in Tabelle ##Tab:Temperatur## zeigen, dass dies hier auszuschließen ist.

Starke Abweichungen könnten dadurch entstehen, wenn die Annahme in \ref{sec:theorie} für das verwandte System ungültig ist, wodurch \ref{eq:redwaemre} und \ref{eq:gueteziffer_ideal} unbrauchbar werden

- "Wie ist das Güteziffer: brauchbar/unbrauchbar?"
- "Welche Fehler könnten auftreten?"
- "Wie kann man das System verbessern?"
- "Eignet sich die großtechnische Anwendung?"