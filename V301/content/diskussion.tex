\section{Diskussion}
\label{sec:Diskussion}
Die Leerlaufspannung $U_0$ der Monozelle wurde auf zwei verschieden Weisen bestimmt. Die direkte Messung $U_0,\mathup{Gl.}=(1.473\pm0.008)\,\si\volt$ mit einem relativen Fehler von $0.5\%$ ist nur bedeutend genauer gegenüber dem Wert mit Gegenspannung $U_0,\mathup{UG}=(1.65\pm0.02)\,\si\volt$ mit einem Fehler von $1.2\%$. 
Die Innenwiderstände weisen relative Fehler von $R_i,\mathup{Gl.}=(5.27\pm0.06)\,\si\ohm$ und $R_i,\mathup{UG}=(5.4\pm0.1)\,\si\ohm$ weisen relative Fehler von $1.2\%$ und $1.9\%$ auf.
Sowohl die Leerlaufspannung, als auch der Innenwiderstand kann anhand beider Methoden sehr genau bestimmt werden.
Gleiches gilt für die Werte, welche mit der Rechteckspannung bestimmt wurden. Diese weisen relative Fehler von $0.94\%$ und $1.72\%$ für Leerlaufspannung und Innenwiderstand auf. Die Messung mit Sinusspannungsquelle ist mit Fehlern von $4.35\%$ und $14.3\%$ etwas ungenauer, liegt aber noch immer im vertretbaren Rahmen.

Die direkte Messung von $U_0$ über das Voltmeter mit endlichem Innenwiderstand liefert einen systematischen Fehler von $(8.2992\cdot10⁻⁷)\,\si{\volt}$, der vernachlässigbar gering ist.

Das Prinzip der Leistungsanpassung besagt, dass die am Belastungswiderstand abfallende Leistung maximal wird, wenn Außen- und Innenwiderstand den gleichen Wert aufweisen.
 Der Vergleich vom Mittelwert $R_\mathup{i}=\SI{5.32\pm0.08}{\ohm}$ mit dem Maximum der Theoriekurve in Abbildung \ref{fig:N}, welches ungefähr bei $R_\mathup{a}=\SI{5}{\ohm}$ liegt, bestätigt die Aussage.

Es zeigt sich, dass mit diesem einfachen Messverfahren alle betrachteten Werte ziemlich genau bestimmt werden können.
