\section{Durchführung}
\label{sec:Durchfuehrung}
\subsection{Freie Schwingung mit Dämpfung}
Im ersten Teil wird die freie, gedämpfte Schwingung mit einem Serienschwingkreis untersucht.
Es wird die Zeitabhängigkeit der Kondensatorspannung $U_\mathup{C}(\omega, t)$ betrachtet und ihr Verlauf diskutiert.
Hierzu wird eine Schaltung wie in Abbildung \ref{fig:schaltkreis_erzwungen} angesetzt.
Der Generator wird auf Rechteckspannung mit einer solch geringen Frequenz $f$ eingestellt, dass innerhalb einer Periode des Generator das System frei oszillieren und abklingen kann, ehe es nach der Periodendauer $\frac{1}{f}$ erneut angeregt wird.
Die weiteren Systemparameter, etwa die Komponenten des Schwingkreises, werden so gewählt, dass die charakteristischen Verläufe sichtbar werden. 
Die Verläufe der Kondensatorspannung $U_\mathup{C}(\omega, t)$ werden dargestellt und gespeichert.

Die systembeschreibende Abklingzeit $T_\text{ex}$ wird an diesen Verläufen bestimmt und mit dem vorhergesagten Wert nach der Gleichung \eqref{eq:abkling} verglichen.\\
Der aperiodische Grenzfall wird als Grenze zwischen Kriechfall und Schwingfall mittels Bisektion des Dämpfungswiderstandes $R$ angenähert, wozu der Schwingkreis mit einem variablen Widerstand ausgestattet wird.
Der auf diese Weise ermittelte Dämpfungswiderstand $R_\text{ap}$ wird mit dem theoretischen Wert $R_\text{ap,t}$ aus \ref{sec:theorie1} verglichen.

\subsection{Erzwungene Schwingung mit Dämpfung}
Im zweiten Teil wird die erzwungene, gedämpfte Schwingung untersucht.
Gemessen wird die Frequenzabhängigkeit der Kondensatorspannung $U_\mathup{C}(\omega, t)$ eines Serienschwingkreises sowie der Phasenwinkel $\phi$ zwischen der Erregerspannung $U(t)$ und der Kondensatorspannung $U_\mathup{C}(\omega, t)$.\\
Hierzu wird einerseits bei verschiedenen Erregerfrequenzen $U(t)$  der Betrag der Kondensatorspannung $U_\mathup{C}(\omega, t)$ und die Zeitdifferenz der Nulldurchgänge von Erregerspannung $U(t)$ und von der Kondensatorspannung $U_\mathup{C}(\omega, t)$ betrachtet.
Weiter wird zur Charakterisierung des Resonanzverhaltens die Güte $q$ und die Resonanzbreite $\mathup{\Delta}\, f$ bestimmt.

% Ein Wanderurlaub im ehemaligen Jugoslawien. Klingt zunächst einmal furchtbar spannend, ist aber eigentlich der Gähner (=Langeweiler, langweilige Sache) überhaupt.

% Jeder denkt: Alte Militärbaracken, Herrenausstatter wohin man schaut, vielleicht ein Museum für altertümliche Fahrstuhltechnik, klingt doch toll! Doch hält diese vorgefasste Meinung einer genaueren Betrachtung nicht stand. Schon morgens im Hotel wird die Kehrseite der Medaille deutlich:

% Aufzug defekt. Buffet unvollständig. Chinesische Loungemusik. Despotisches Hotelpersonal. Eierlikör ausverkauft. Französischer Kofferträger. Gewaltsamer Raubüberfall. Hasenzähnige Empfangsdame. Interplanetarer Schmugglerstützpunkt. Jodelmusikkorps nebenan. Kreditkarte gesperrt. Lilafarbener Teppichläufer. Monochromatisches Licht. Nucleophile Substitution. Ortsunkundige Japaner. Präsidentenleiche unübersehbar. Qualmender Ethanolofen. Resistiver Touchscreen. Systematische Tötungen. Trauriger Clown. Unerfreuliche Massenbegräbnisse. Verwanzte Matratzen. Wadenkrampffördernde Beleuchtung. X-Beinige Pianodame. Yorkshireterrier bellt. Bezahlung nur bar möglich (und nicht per EC-Karte, wie ich es sonst immer mache).

% Also merke: Der Spruch „Im Norden geht die Sonne auf, im Süden nimmt sie Ihren Lauf, […] Westen wird sie untergehen, usw.“, gilt nicht wenn man auf dem Mond (oder einem anderen Erdtrabanten) steht (oder sitzt, außer man liegt).
