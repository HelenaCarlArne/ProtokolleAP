
\section{Auswertung}
\label{sec:auswertung}
\subsection{Verifikation der Linsengleichung}
\label{sec:auswertung1}
Die Ergebnisse der ersten Messung sind in Tabelle \ref{tab:M1} aufgetragen.
\begin{figure}[hp]
	\begin{minipage}{0.49\textwidth}
		\centering
		\begin{tabular}{S[table-format=3] S[table-format=3] S[table-format=3.2]}
			\toprule
			\multicolumn{3}{c}{Linse mit $\tilde{f}=\SI{100}{\milli\meter}$}\\
			%\multicolumn{3}{c}{f_1=100\,\si{\milli\meter}} & \multicolumn{3}{c}{f_2=50\,\si{\milli\meter}} \\
			{$g_1/\:\si{\milli{\meter}}$} & {$b_1/\:\si{\milli{\meter}}$} & {$f_1/\:\si{\milli{\meter}}$} \\	
			\midrule
			120 & 525 & 97.67\\
			130 & 390 & 97.50\\
			140 & 319 & 97.30\\
			150 & 277 & 97.31\\
			160 & 251 & 97.71\\
			170 & 227 & 97.20\\
			180 & 204 & 95.63\\
			190 & 198 & 96.96\\
			200 & 192 & 97.59\\
			210 & 186 & 98.64\\
			220 & 177 & 98.09\\
			\bottomrule
			\end{tabular}
	\end{minipage}
	\begin{minipage}{0.49\textwidth}
		%\centering
		\begin{tabular}{S[table-format=3] S[table-format=3] S[table-format=3.2]}
			\toprule
			\multicolumn{3}{c}{Linse mit $\tilde{f}=\SI{50}{\milli\meter}$}\\
			%\multicolumn{3}{c}{f_1=100\,\si{\milli\meter}} & \multicolumn{3}{c}{f_2=50\,\si{\milli\meter}} \\
			{$g_2/\:\si{\milli{\meter}}$} & {$b_2/\:\si{\milli{\meter}}$} & {$f_2/\:\si{\milli{\meter}}$}\\	
			\midrule
				60  & 270 & 49.1 \\
				70  & 157 & 48.4 \\
				80  & 121 & 48.2 \\
				90  & 104 & 48.2 \\
				100 &  92 & 47.9 \\
				110 &  87 & 48.6 \\
				120 &  80 & 48.0 \\
				130 &  77 & 48.4 \\
				140 &  75 & 48.8 \\
				150 &  72 & 48.2 \\
			\bottomrule
			\end{tabular}
		\end{minipage}
	\caption{Messung der Bildweiten $b_i$ bei festgelegter Gegenstandsweite $g_i$ sowie die daraus berechneten Brennweiten nach der Linsengleichung.}
	\label{tab:M1}
\end{figure}
Für die berechnete Brennweite der Linsen ergeben sich Werte von 
\begin{align}
	f_1 &= \SI{97.5(2)}{\milli\meter}\\
	f_2 &= \SI{48.4(1)}{\milli\meter}.
\end{align}
Das $b$-$g$-Diagramm \ref{fig:bgdiagramm} zeigt dadurch, dass sich die Linien auf einem kleinen, nahezu punktförmigen Gebiet untereinander schneiden, die verhältnismäßig hohe Präzession der Messergebnisse an.
Die Mittelwerte weichen von der Herstellerangabe um 
\begin{align}
	\mathup{\Delta}f_1 &= 2.5\% \quad\text{und}\quad\mathup{\Delta}f_2 = 3.2\%
\end{align}
ab.
Daher ist für die verwendeten Linsen die Brennweite $f$ über die Linsengleichung \eqref{eq:linsengleichung} verifizierbar.
\begin{figure}[hb] %% b-g-Diagramm
	\centering
	\begin{subfigure}{0.9\textwidth}
	\includegraphics[width=\textwidth]{Bilder/Messung1.pdf}
	\end{subfigure}
	\begin{subfigure}{0.9\textwidth}
	\includegraphics[width=\textwidth]{Bilder/Messung2.pdf}
	\end{subfigure}
	\caption{$b$-$g$-Diagramme zur Darstellung der Messgenauigkeit. \cite{matplotlib}}
	\label{fig:bgdiagramm} 
\end{figure}
\subsection{Methode nach \texorpdfstring{\textsc{Bessel}}{Bessel}}
\label{sec:auswertung2}
Die Ergebnisse der Messung nach dem \texorpdfstring{\textsc{Bessel}}{Bessel}-Verfahren sind in Tabelle \ref{tab:M2} aufgetragen.
\begin{table}[htp]
	\centering
	\begin{tabular}{S[table-format=3.0] S[table-format=2.1] S[table-format=2.1] S[table-format=3.2] S[table-format=2.1] S[table-format=2.1] S[table-format=3.2]}
	\toprule
	{Abstand} &\multicolumn{3}{c}{Linsenposition 1} & \multicolumn{3}{c}{Linsenposition 2} \\
		{$e/\:\si{\milli{\meter}}$} & {$g_1/\:\si{\milli{\meter}}$} & {$b_1/\:\si{\milli{\meter}}$} & {$f_1/\:\si{\milli{\meter}}$} & {$g_2/\:\si{\milli{\meter}}$}  & {$b_2/\:\si{\milli{\meter}}$} & {$f_2/\:\si{\milli{\meter}}$}\\	
		\midrule
		450 & 144 & 306 & 97.9 &  142 & 308 & 97.2 \\
		500 & 134 & 366 & 98.1 &  135 & 365 & 98.6 \\
		550 & 127 & 423 & 97.7 &  127 & 423 & 97.7 \\
		600 & 122 & 478 & 97.2 &  123 & 477 & 97.8 \\
		650 & 119 & 531 & 97.2 &  120 & 530 & 97.8 \\
		700 & 118 & 482 & 127.7 & 117 & 583 & 97.4 \\
		750 & 116 & 734 & 60.2 &  115 & 735 & 59.4 \\
		800 & 114 & 688 & 97.8 &  116 & 688 & 99.1 \\
		850 & 114 & 736 & 98.7 &  114 & 736 & 98.7 \\
		900 & 113 & 787 & 98.8 &  113 & 788 & 98.4 \\
			\bottomrule
		\end{tabular}
	\caption{Messung der Projektionsweiten $b_i$ und $g_i$ bei festgelegtem Abstand $e$ nach Bessel; weißes Licht.}
	\label{tab:M2}  %% Weißes Licht, Bessel
\end{table}
Für die Brennweiten ergeben sich Werte für die beiden Linsenpositionen von 
\begin{alignat}{3}
	f-\text{Pos.1} = \SI{97(5)}{\milli\meter} \quad\text{und} \quad f_\text{Pos.2}= \SI{94(4)}{\milli\meter}.
\end{alignat}
Die gemessene Brennweite weicht in Abhängigkeit von der Linsenposition geringfügig ab,
die Schwankungen sind als statistische Fehler zu bewerten.
Der Mittelwert
\begin{equation}
	f = \SI{96(3)}{\milli\meter}
\end{equation}
zeigt eine Abweichung von der Herstellerangabe von $4\%$.

Die Ergebnisse der Messung mit einfarbigem Licht sind in den Tabellen \ref{tab:M2a} und \ref{tab:M2b} aufgetragen.
Die ermittelten Brennweiten in Abhängigkeit von der Linsenposition betragen
\begin{subequations}
	\begin{align}
		f_\text{Rot, Pos.1} &= \SI{115(1)}{\milli\meter}\\
		f_\text{Rot, Pos.2} &= \SI{114.1(7)}{\milli\meter}\\
		f_\text{Rot} &= \SI{114.4(7)}{\milli\meter}
	\end{align}
	\begin{align}
		f_\text{Blau, Pos.1} &= \SI{98.2(2)}{\milli\meter}\\
		f_\text{Blau, Pos.2} &= \SI{97.2(2)}{\milli\meter}\\
		f_\text{Blau} &= \SI{97.7(2)}{\milli\meter}
	\end{align}
\end{subequations}
und zeigen damit die Abhängigkeit der Brennweite von der Wellenlänge des Lichtes.
Auch hier wird sichtbar, dass die errechneten Brennweiten von der Linsenposition abhängen; 
die Schwankungen sind aber als statistische Fehler zu bewerten.
\begin{table}[p]
		\centering
		\begin{tabular}{S[table-format=2.0] S[table-format=3.0] S[table-format=3.0] S[table-format=3.2] S[table-format=3.0] S[table-format=3.0] S[table-format=3.2] }
		\toprule
			{Abstand}&\multicolumn{3}{c}{Linsenposition 1} & \multicolumn{3}{c}{Linsenposition 2} \\
			{$e/\:\si{\milli{\meter}}$} & {$g_{1,\mathup{r}}/\:\si{\milli{\meter}}$} & {$b_{1,\mathup{r}}/\:\si{\milli{\meter}}$} & {$f_{1,\mathup{r}}/\:\si{\milli{\meter}}$} & {$g_{2,\mathup{r}}/\:\si{\milli{\meter}}$} & {$b_{2,\mathup{r}}/\:\si{\milli{\meter}}$} & {$f_{2,\mathup{r}}/\:\si{\milli{\meter}}$} \\	
			\midrule
			50 & 143 & 307 & 111.6 & 306 & 144 & 111.9  \\
			60 & 126 & 424 & 113.0 & 424 & 126 & 113.0 \\
			70 & 118 & 532 & 113.8 & 531 & 119 & 114.4 \\
			80 & 117 & 633 & 116.8 & 636 & 117 & 115.8 \\
			90 & 115 & 735 & 118.2 & 739 & 111 & 115.5 \\
			\bottomrule
			\end{tabular}
			\caption{Messung der Projektionsweiten $b_i$ und $g_i$ bei festgelegtem Abstand $e$ nach Bessel; rotes Licht.}
			\label{tab:M2a} %% Farbiges Licht, Bessel
\end{table}
\begin{table}[p]
			\centering
			\begin{tabular}{S[table-format=2.0] S[table-format=3.0] S[table-format=3.0] S[table-format=3.2] S[table-format=3.0] S[table-format=3.0] S[table-format=3.2] }
			\toprule	
				{Abstand} &\multicolumn{3}{c}{Linsenposition 1} & \multicolumn{3}{c}{Linsenposition 2} \\
				{$e/\:\si{\milli{\meter}}$} & {$g_{1,\mathup{b}}/\:\si{\milli{\meter}}$} & {$b_{1,\mathup{b}}/\:\si{\milli{\meter}}$} & {$f_{1,\mathup{b}}/\:\si{\milli{\meter}}$} &{$g_{2,\mathup{b}}/\:\si{\milli{\meter}}$}  & {$b_{2,\mathup{b}}/\:\si{\milli{\meter}}$} & {$f_{2,\mathup{b}}/\:\si{\milli{\meter}}$}\\	
				\midrule
				50 & 366 & 134 & 98.1 & 132 & 368 & 97.2\\
				60 & 477 & 123 & 97.8 & 122 & 478 & 97.2\\
				70 & 582 & 118 & 98.1 & 116 & 584 & 96.8\\
				80 & 684 & 116 & 99.1 & 114 & 686 & 97.8\\
				90 & 788 & 112 & 98.1 & 111 & 789 & 97.3\\
				\bottomrule
			\end{tabular}
			\caption{Messung der Projektionssweiten $b_i$ und $g_i$ bei festgelegtem Abstand $e$ nach Bessel; blaues Licht.}
			\label{tab:M2b}
\end{table}

\subsection{Methode nach \texorpdfstring{\textsc{Abbe}}{Abbe}}
\label{sec:auswertung3}
\begin{table}[h]
	\centering
	\begin{tabular}{S[table-format=3.0] S[table-format=3.0] S[table-format=2.0] S[table-format=1.2]}
	\toprule
	\multicolumn{4}{c}{Linsensystem}\\
		 {$g'/\:\si{\milli\meter}$} & {$b'/\:\si{\milli{\meter}}$} & {$B/\:\si{\milli{\meter}}$} & {$V/\:\si{\milli{\meter}}$}\\	
		\midrule
		200 & 790 & 80 & 2.67\\
		250 & 551 & 44 & 1.47\\
		300 & 480 & 31 & 1.03\\
		350 & 416 & 25 & 0.83\\
		400 & 398 & 20 & 0.67\\
		450 & 380 & 17 & 0.57\\
		500 & 370 & 15 & 0.50\\
		550 & 346 & 13 & 0.43\\
		600 & 348 & 11 & 0.37\\
		650 & 336 & 11 & 0.37\\
	\bottomrule
	\end{tabular}
	\caption{Messwerte zur Bestimmung der Brennweite des Linsensystems nach Abbe.} %% Abbe
	\label{tab:M3}
\end{table}

Die Linearisierung der Gleichungen
\begin{subequations}
\begin{equation}
	\underbrace{g'}_{y_\text{lin}}=\underbrace{f}_{m_\text{lin}}\cdot\underbrace{\biggl(1+\frac{1}{V}\biggr)}_{x_\text{lin}}+\underbrace{h}_{b_\text{lin}}
	\label{eq:abbe1}
	\end{equation}
	\begin{equation}
	\underbrace{b'}_{y_\text{lin}}=\underbrace{f}_{m_\text{lin}}\cdot\underbrace{\bigl(1+V\bigr)}_{x_\text{lin}}+\underbrace{h'}_{b_\text{lin}}
	\label{eq:abbe2}
\end{equation}
\end{subequations}
mit den Werten der Tabelle \ref{tab:M3} sind in Abbildung \ref{fig:abbe1} und \ref{fig:abbe2} dargestellt.
Die Regression mithilfe der Formeln
 \begin{figure}[h]
\centering
\begin{subequations}
	\begin{equation}
		\Delta = N \sum{x^2} - {\biggl(\sum{x}\biggr)}^2,
	\end{equation}
	\begin{equation}
		a_{\text{Reg}} = \frac{N\sum{x\cdot y} - \sum{x} \cdot \sum{y}}{\Delta},
	\end{equation}
    \begin{equation}
		b_{\text{Reg}} = \frac{\sum{x^2} \cdot \sum{y} - \sum{x} \cdot \sum{x \cdot y}}{\Delta},
	\end{equation}
	\begin{equation}
		\sigma_{y} = \sqrt{\frac{\sum{(y - a_{\text{Reg}} \cdot x - b_{\text{Reg}})^2}}{N - 2}},
	\end{equation}
	\begin{equation}
		\sigma_{a} = \sigma_{y} \sqrt{\frac{N}{\Delta}},
	\end{equation}
	\begin{equation}
		\sigma_{b} = \sigma_{y} \sqrt{\frac{\sum{x^2}}{\Delta}}
	\end{equation}
	%mit $x=x_\text{lin}=\frac{1}{T_m^2}$, $y=B$, $a_\text{Reg}=a_\text{lin}$, $b_\text{Reg}=b_\text{lin}$ und der Anzahl der Datenpaare N.
	\label{eq:regress}
\end{subequations}
\end{figure}
mit den in Gleichung \eqref{eq:abbe2} definierten Abkürzungen und der Anzahl der Datenpaare N, ergibt
\begin{alignat}{3}
	h_1 = \SI{55(14)}{\milli\meter} & \qquad&&h_2 = \SI{72(7)}{\milli\meter}
\end{alignat}
\begin{equation}
	f = \SI{189(5)}{\milli\meter}
\end{equation}


\begin{figure}[p]
	\centering
	\includegraphics[width=0.8\textwidth]{Bilder/Abbe1.pdf}
	\caption{Messwerte für \texorpdfstring{\textsc{Abbe}}{Abbe}-Methode und Regression der Gleichung \eqref{eq:abbe1}. \cite{matplotlib}}
	\label{fig:abbe1}
\end{figure}
\begin{figure}[p]
	\centering
	\includegraphics[width=0.8\textwidth]{Bilder/Abbe2.pdf}
	\caption{Messwerte für \texorpdfstring{\textsc{Abbe}}{Abbe}-Methode und Regression der Gleichung \eqref{eq:abbe2}.\cite{matplotlib}}
	\label{fig:abbe2}
\end{figure}