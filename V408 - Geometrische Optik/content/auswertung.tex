
\section{Auswertung}
\label{sec:Auswertung}
\subsection{Verifikation der Linsengleichung}
Die Ergebnisse der ersten Messung sind in Tabelle \ref{tab:M1} aufgetragen.
\begin{table}
	\centering
	\begin{tabular}{S[table-format=3] S[table-format=3] S[table-format=3.2]|S[table-format=3] S[table-format=3] S[table-format=3]}
		\toprule
		%\multicolumn{3}{c}{f_1=100\,\si{\milli\meter}} & \multicolumn{3}{c}{f_2=50\,\si{\milli\meter}} \\
		{$g_1/\:\si{\milli{\meter}}$} & {$b_1/\:\si{\milli{\meter}}$} & {$f_1/\:\si{\milli{\meter}}$} & {$g_2/\:\si{\milli{\meter}}$} & {$b_2/\:\si{\milli{\meter}}$} & {$f_2/\:\si{\milli{\meter}}$}\\	
		\midrule
		120 & 525 & 98.09 &     &      &   	\\
		130 & 390 & 98.64 &  60 & 2700 & 124.14 \\
		140 & 319 & 97.59 &  70 & 1570 & 117.98 \\
		150 & 277 & 96.96 &  80 & 1210 & 111.22 \\
		160 & 251 & 95.63 &  90 & 1040 & 104.35 \\
		170 & 227 & 97.20 & 100 &  920 &  97.65 \\
		180 & 204 & 97.71 & 110 &  870 &  90.20 \\
		190 & 198 & 97.31 & 120 &  800 &  82.83 \\
		200 & 192 & 97.30 & 130 &  770 &  75.04 \\
		210 & 186 & 97.50 & 140 &  750 &  67.01 \\
		220 & 177 & 97.67 & 150 &  720 &  58.70 \\
		\bottomrule
		\end{tabular}
	\caption{Messung der Bild- und Gegenstandsweiten $b_i$ und $g_i$, sowie die daraus berechneten Brennweiten.}
	\label{tab:M1}
\end{table}
Für die berechnete Brennweite ergibt sich ein Wert von 
\begin{equation}
	f = dsahjkdaskhj\pm kldasasdkle
\end{equation}
Dies weicht von der Herstellerangabe um \# ab, die Standardabweichung zeigt eine starke Schwankung an.

\subsection{Methode nach Bessel}
Die Ergebnisse der Messung nach dem Bessel-Verfahren sind in Tabelle \ref{tab:M2} aufgetragen.
\begin{table}
	\centering
	\sisetup{table-format=3.2}
	\begin{tabular}{S S S S S S S }
	\toprule
	%\multicolumn{3}{c}{f_1=100\,\si{\milli\meter}} & \multicolumn{3}{c}{f_2=50\,\si{\milli\meter}} \\
		{$e/\:\si{\milli{\meter}}$} & {$g_1/\:\si{\milli{\meter}}$} & {$b_1/\:\si{\milli{\meter}}$} & {$f_1/\:\si{\milli{\meter}}$} & {$g_2/\:\si{\milli{\meter}}$}  & {$b_2/\:\si{\milli{\meter}}$} & {$f_2/\:\si{\milli{\meter}}$}\\	
		\midrule
		450 & 14.4 & 30.6 & 112.36 & 14.2 & 30.8 & 112.35 \\
		500 & 13.4 & 36.6 & 124.73 & 13.5 & 36.5 & 124.74 \\
		550 & 12.7 & 42.3 & 137.10 & 12.7 & 42.3 & 137.10 \\
		600 & 12.2 & 47.8 & 149.47 & 12.3 & 47.7 & 149.48 \\
		650 & 11.9 & 53.1 & 161.85 & 12.0 & 53.0 & 161.85 \\
		700 & 11.8 & 48.2 & 174.53 & 11.7 & 58.3 & 174.24 \\
		750 & 11.6 & 73.4 & 186.23 & 11.5 & 73.5 & 186.22 \\
		800 & 11.4 & 68.8 & 198.98 & 11.6 & 68.8 & 198.99 \\
		850 & 11.4 & 73.6 & 211.36 & 11.4 & 73.6 & 211.36 \\
		900 & 11.3 & 78.7 & 223.74 & 11.3 & 78.8 & 223.73 \\
			\bottomrule
		\end{tabular}
	\caption{Messung der Bild- und Gegenstandsweiten $b_i$ und $g_i$ nach Bessel in Abhängigkeit vom Abstand $e$.}
	\label{tab:M2}
\end{table}
Für die berechnete Brennweite ergibt sich ein Wert von 
\begin{equation}
	f = dsahjkdaskhj\pm kldasasdkle
\end{equation}
Dies weicht von der Herstellerangabe um \# ab.
Das $b$-$g$-Diagramm \ref{fig:bgdiagramm} zeigt dadurch, dass sich die Linien auf einem nicht punktförmigen Gebiet untereinander schneiden, starke Unsicherheit der Messergebnisse an.

Die Ergebnisse der Messung mit einfarbigem Licht sind in den Tabellen \ref{tab:M2a} und \ref{tab:M2b} aufgetragen.
Die ermittelten Brennweiten betragen
\begin{align}
	f_\text{Rot} &= dsahjkdaskhj\pm kldasasdkle\\
	f_\text{Blau} &= dsahjkdaskhj\pm kldasasdkle
\end{align}
und zeigen damit die Abhängigkeit der Brechung von der Wellenlänge des Lichtes.
\begin{table}
	\centering
	\sisetup{table-format=3.2}
	\begin{tabular}{S S S S S S S }
	\toprule
		{$e/\:\si{\milli{\meter}}$} & {$g_{1\mathup{r}}/\:\si{\milli{\meter}}$} & {$b_{1\mathup{r}}/\:\si{\milli{\meter}}$} & {$f_{1\mathup{r}}/\:\si{\milli{\meter}}$} & {$g_{2\mathup{r}}/\:\si{\milli{\meter}}$} & {$b_{2\mathup{r}}/\:\si{\milli{\meter}}$} & {$f_{2\mathup{r}}/\:\si{\milli{\meter}}$} \\	
		\midrule
		50 & 143 & 307 & 111.88 & 306 & 144 & 111.89  \\
		60 & 126 & 424 & 113.00 & 424 & 126 & 113.00 \\
		70 & 118 & 532 & 114.38 & 531 & 119 & 114.38 \\
		80 & 117 & 633 & 115.82 & 636 & 117 & 115.82 \\
		90 & 115 & 735 & 115.45 & 739 & 111 & 115.45 \\
			\bottomrule
		\end{tabular}
			\caption{Messung der Bild- und Gegenstandsweiten $b_{i\mathup{r}}$ und $g_{i\mathup{r}}$ von rotem Licht nach Bessel in Abhängigkeit vom Abstand $e$.}
			\label{tab:M2a}
\end{table}

\begin{table}
		\centering
		\sisetup{table-format=3.2}
		\begin{tabular}{S S S S S S S}
		\toprule
			{$e/\:\si{\milli{\meter}}$} & {$g_{1\mathup{b}}/\:\si{\milli{\meter}}$} & {$b_{1\mathup{b}}/\:\si{\milli{\meter}}$} & {$f_{1\mathup{b}}/\:\si{\milli{\meter}}$} &{$g_{2\mathup{b}}/\:\si{\milli{\meter}}$}  & {$b_{2\mathup{b}}/\:\si{\milli{\meter}}$} & {$f_{2\mathup{b}}/\:\si{\milli{\meter}}$}\\	
			\midrule
			50 & 366 & 134 & 97.15 & 132 & 368 & 97.15 \\
			60 & 477 & 123 & 97.19 & 122 & 478 & 97.19\\
			70 & 782 & 118 & 15.63 & 116 & 784 & 15.63\\
			80 & 784 & 116 & 58.88 & 114 & 786 & 58.89\\
			90 & 788 & 112 & 97.31 & 111 & 789 & 97.31\\
			\bottomrule
		\end{tabular}
		\caption{Messung der Bild- und Gegenstandsweiten $b_{i\mathup{b}}$ und $g_{i\mathup{b}}$ von blauem Licht nach Bessel in Abhängigkeit vom Abstand $e$.}
		\label{tab:M2b}
\end{table}

\subsection{Methode nach Abbe}
\begin{table}
	\centering
	\sisetup{table-format=3.2}
	\begin{tabular}{S S S S}
	\toprule
		 {$g'/\:\si{\milli{\meter}}$} & {$b'/\:\si{\milli{\meter}}$} & {$B/\:\si{\milli{\meter}}$} & {$V/\:\si{\milli{\meter}}$}\\	
		\midrule
		200 & 790 & 80 & 2.67\\
		250 & 551 & 44 & 1.47\\
		300 & 480 & 31 & 1.03\\
		350 & 416 & 25 & 0.83\\
		400 & 398 & 20 & 0.67\\
		450 & 380 & 17 & 0.57\\
		500 & 370 & 15 & 0.50\\
		550 & 346 & 13 & 0.43\\
		600 & 348 & 11 & 0.37\\
		650 & 336 & 11 & 0.37\\
	\bottomrule
	\end{tabular}
	\caption{Messwerte zur Bestimmung der Brennweite des Linsensystems nach \textsc(abbe).}
\end{table}
