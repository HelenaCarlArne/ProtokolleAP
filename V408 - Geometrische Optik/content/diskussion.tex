\section{Diskussion}
\label{sec:Diskussion}
\subsection{Fehlerdiskussion}
Über die einzelnen Teile des Experimentes hinweg wird eine scharfe Abbildung des Gegenstandes "Perl L" gefordert.
Die in Abschnitt \ref{sec:theorie} genannten Abbildungsfehler, 
insbesondere die in Abschnitt \ref{sec:auswertung2} bestätigte chromatische Abberation, 
erschweren das Finden der richtigen Projektionsweiten. 
Das exakte Bestimmen der Projektionsweiten ist ohne weitere Maßnahmen oder geräte-unterstützte Messung, etwa durch einen CCD-Chip, nicht möglich.
Durch Bisektion kann die Größenordnung und die Umgebung von $b$ und $g$ bestimmt werden; es treten dabei starke Unsicherheiten auf.
Im Abschnitt \ref{sec:auswertung1} wird die Unsicherheit in $b$ und $g$ besonders durch das Diagramm \ref{fig:bgdiagramm} erkennbar.

\subsection{Linsengleichung}
Der erste Abschnitt der Messung bestätigt die Gültigkeit der Linsengleichung.\\
Wegen der starken Abweichung der zweiten Linse im Konstrast zur ersten Linse muss das Auftreten von systematischen Fehlern diskutiert werden.

\subsection{Methode nach Bessel}
Die Methode von Bessel kann mit der konventionellen Methode über die Linse 1 verglichen werden.\\
Die Abweichung des Mittelwertes von der Herstellerangabe ist ein direktes Maß für die Fehleranfälligkeit der Methoden. 
Es ist erkennbar, dass die Methode nach Bessel für die in diesem Experiment durchgeführte Bestimmung der konventionelle Methode unterlegen ist.

Für die Methode nach Bessel werden der Abstand von Gegenstand und Schirm sowie die Differenz der Projektionslängen benötigt. 
Da diese in abgewandelter Form ebenfalls für die Referenzmethode gilt, kann von systematischen Fehlern im Verlauf der Messung ausgegangen werden.

\subsection{Methode nach Abbe}