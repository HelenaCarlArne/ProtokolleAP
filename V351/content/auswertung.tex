\section{Auswertung}
\label{sec:Auswertung}
\subsection{Fourieranalyse}

\begin{table}
	\centering
	\begin{tabular}{S[table-format=2.0] S[table-format=3.0] S[table-format=3.0] S[table-format=2.2]}
	\toprule
{$n$} & {$U_n/\:\si{\milli{\volt}}$} & {$\frac{U_1}{n}/\:\si{\milli{\volt}}$} & {Abweichung $\:/\%$}\\
	\midrule
 1 & 920 & 912 &  1\\
 3 & 300 & 288 &  4\\
 5 & 170 & 168 &  1\\
 7 & 116 & 112 &  4\\
 9 &  86 &  80 &  8\\
11 &  64 &  56 & 14\\
	\bottomrule
	\end{tabular}
	\caption{Fourieranalyse der Rechteckspannung.}
	\label{tab:FA_RE}
\end{table}
 
\begin{table}
	\centering
	\sisetup{table-format=2.3}
	\begin{tabular}{S[table-format=2.0] S[table-format=3.0] S[table-format=3.1] S[table-format=1.2] }
	\toprule
	{$n$} & {$U_n/\:\si{\milli\volt}$} & {$\frac{U_1}{n}/\:\si{\milli\volt}$} & {Abweichung $\:/\%$}\\
	\midrule
 1 & 580 & 576.0 & 0.7\\
 3 &  63 &  61.6 & 2.3\\
 5 &  21 &  20.8 & 1.0\\
 7 &  10 &  10.4 & 3.9\\
 9 &   6 &   5.6 & 7.1\\
11 &   4 &   4.0 & 0.0\\
	\bottomrule
	\end{tabular}
	\caption{Fourieranalyse der Dreiecksspannung.}
	\label{tab:FA_DE}
\end{table}



\begin{table}
	\centering
	\sisetup{table-format=2.3}
	\begin{tabular}{S[table-format=2.0] S[table-format=3.0] S[table-format=3.1] S[table-format=1.0] }
	\toprule
	{$n$} & {$U_n/\:\si{\milli{\volt}}$} & {$\frac{U_1}{n}/\:\si{\milli\volt}$} & {Abweichung $\:/\%$}\\
	\midrule
 1 & 460 & 456 & 1\\
 2 & 230 & 224 & 3\\
 3 & 152 & 144 & 6\\
 4 & 112 & 112 & 0\\
 5 &  88 &  88 & 0\\
 6 &  74 &  74 & 0\\
 7 &  64 &  64 & 0\\
 8 &  56 &  56 & 0\\
 9 &  50 &  48 & 4\\
10 &  44 &  46 & 5\\
11 &  40 &  40 & 0\\
	\bottomrule
	\end{tabular}
	\caption{Fourieranalyse der Sägezahnspannung.}
	\label{tab:FA_SZ}
\end{table}

\subsection{Fouriersynthese}


\begin{table}
	\centering
	\begin{tabular}{cccc}	
	\toprule
\multicolumn{1}{c}{n} & \multicolumn{3}{c}{$U_n\:/\si{\milli{\volt}}$}\\
	{} & {Rechteck} & {Sägezahn} & {Dreieck}\\
	\midrule
 1 & 634.8  & 634.8  & 634.80\\
 2 &   0    & 318.00 &   0\\
 3 & 216.73 & 212.00 &  70.72\\
 4 &   0    & 159.76 &   0\\
 5 & 130.20 & 127.65 &  25.48\\
 6 &   0    & 106.45 &   0\\
 7 &  92.90 &  91.20 &  13.27\\
 8 &   0    &  79.49 &   0\\
 9 &  72.08 &  70.55 &  7.95\\
	\bottomrule
	\end{tabular}
	\caption{Fouriersynthese drei verschiedener Spannungen.}
	\label{tab:FS}
\end{table}

