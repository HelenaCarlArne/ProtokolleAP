\section*{Ziel}
Ziel des Versuchs ist es, die Fourier-Transformation kennenzulernen. 
Hierzu wird zum Einen eine bekannte periodische Funktion durch Fourier-Transformation in die Elementarschwingung zerlegt und 
zum Anderen eine periodische Funktion aus Elementarschwingungen gebildet.
\section{Theorie}
\label{sec:Theorie}

Für eine T-periodische Funktion $f$ der Zeit gilt
\begin{equation}
	f(t) = f(t+\mathup{T}) \forall t, 
\end{equation}
für eine Q-periodische Funktion $g$ des Ortes gilt
\begin{equation}
	g(x) = g(x+Q) \forall x. 
\end{equation}
Nach dem Fourier'schen Theorem lassen sich solche periodischen Funktionen, etwa Wellen, als Linearkombination aus den Elementarschwingungen
\begin{alignat}{3}
	a_\text{n}\text{sin}(\frac{2π}{\mathup{T}}t) &\text{und} && b_\text{n}\text{cos}(\frac{2π}{\mathup{T}}t) .
\end{alignat}