\section{Zielsetzung}

Versuchsziel ist es, sich mit der Funktionsweise des Lock-In-Verstärkers vertraut zu machen. Außerdem soll für 10 verschiedene Phasen mit und ohne verrauschtes Signal die Funktionsweise verifiziert werden. Zuletzt soll die Rauschunterdrückung mit einer Photodetektorschaltung überprüft werden.

\section{Theorie}
\label{sec:Theorie}

Lock-In-Verstärker werden eingesetzt, um Signale mit hohem Rauschen zu messen. 
Im Gegensatz zum Bandpass kann hier auch Rauschen herausgefiltert werden, welches auf der selben Frequenz wie das Messsignal liegt.
%Dafür wird das Messsignal mit einer Referenzfrequenz $\omega_0$ moduliert.

Das zu messende Eingangssignal $U_\mathup{sig}$ durchläuft im Gerät verschiedene Bauelemente.

Nach der Verstärkung durch den Pre-Amplifier durchläuft das Signal zunächst einen Bandpassfilter, der das Rauschen minimiert. 
Alle Frequenzen $\omega$. die nicht der Referenzfrequenz $\omega_0$ entsprechen werden grob herausgefiltert.
Ein Detektor erzeugt die Referenzspannung $U_\mathup{ref}$ - eine Sinus- oder Rechteckspannung - der Frequenz $\omega_0$, welche über den Phasenschieber an die Phase des Eingangssignals angepasst wird. 
Dieser Vorgang nennt sich Synchronisation.
Im Mischer treffen beide Signale aufeinander, werden multipliziert und anschließend an den Tiefpass weitergeleitet.
Der Tiefpass funktioniert als Integrierer. 
Die Modulationsfrequenz $\omega_0$ wird über mehrere Perioden integriert um restliche Rauschanteile auszuschließen. Zurück bleiben nur die Anteile der Signalsspannung $U_\mathup{sig}$, die mit der Referenzspannung synchronisiert werden konnten.

Um eine möglichst geringe Bandbreite $\Delta{\nu}=\frac{1}{\pi RC}$ zu erhalten, sollte die Zeitkonstante $\uptau=RC$ ausreichend groß gewählt werden. Damit wird eine hohe Güteziffer erzielt.

Die Ausgangsspannung $U_\mathup{out}$ ist proportional zur Eingangsspannung und zum COSINUS . Je größer die Phasendifferenz zwischen Signal- und Referenzspannung ist, desto geringer ist die Ausgangsspannung. $U_\mathup{out}$ wird also maximal, wenn die Phasendifferenz $\Delta\Phi=0$ beträgt.
http://www.physik.uni-regensburg.de/studium/praktika/a2/download/versuch5a.pdf



