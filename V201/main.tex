\documentclass[
  bibliography=totoc,     % Literatur im Inhaltsverzeichnis
  captions=tableheading,  % Tabellenüberschriften
  titlepage=firstiscover, % Titelseite ist Deckblatt
]{scrartcl}

% LaTeX2e korrigieren.
\usepackage{fixltx2e}
% Warnung, falls nochmal kompiliert werden muss
\usepackage[aux]{rerunfilecheck}

% deutsche Spracheinstellungen
\usepackage{polyglossia}
\setmainlanguage{german}

% unverzichtbare Mathe-Befehle
\usepackage{amsmath}
% viele Mathe-Symbole
\usepackage{amssymb}
% Erweiterungen für amsmath
\usepackage{mathtools}

% Fonteinstellungen
\usepackage{fontspec}
\defaultfontfeatures{Ligatures=TeX}

\usepackage[
  math-style=ISO,    % \
  bold-style=ISO,    % |
  sans-style=italic, % | ISO-Standard folgen
  nabla=upright,     % |
  partial=upright,   % /
]{unicode-math}

% Warnung! Bei Aktivierung der alternativen mathfonts (nächsten drei Befehle) 
% könnte die math-Umgebung nicht mehr funktionieren. -Arne

%\setmathfont[range={\mathscr, \mathbfscr}]{XITS Math}
%\setmathfont[range=\coloneq]{XITS Math}
%\setmathfont[range=\propto]{XITS Math}
% make bar horizontal, use \hslash for slashed h
\let\hbar\relax
\DeclareMathSymbol{\hbar}{\mathord}{AMSb}{"7E}
\DeclareMathSymbol{ℏ}{\mathord}{AMSb}{"7E}

% richtige Anführungszeichen
\usepackage[autostyle]{csquotes}

% Zahlen und Einheiten
\usepackage[
  locale=DE,                   % deutsche Einstellungen
  separate-uncertainty=true,   % Immer Fehler mit \pm
  per-mode=symbol-or-fraction, % m/s im Text, sonst Brüche
]{siunitx}

% chemische Formeln
\usepackage[version=3]{mhchem}

% schöne Brüche im Text
\usepackage{xfrac}

% Floats innerhalb einer Section halten
\usepackage[section, below]{placeins}
% Captions schöner machen.
\usepackage[
  labelfont=bf,        % Tabelle x: Abbildung y: ist jetzt fett
  font=small,          % Schrift etwas kleiner als Dokument
  width=0.9\textwidth, % maximale Breite einer Caption schmaler
]{caption}
% subfigure, subtable, subref
\usepackage{subcaption}

% Grafiken können eingebunden werden
\usepackage{graphicx}
% größere Variation von Dateinamen möglich
\usepackage{grffile}

% Standardplatzierung für Floats einstellen
\usepackage{float}
\floatplacement{figure}{h}
\floatplacement{table}{h}

% schöne Tabellen
\usepackage{booktabs}

% Seite drehen für breite Tabellen
\usepackage{pdflscape}

% Literaturverzeichnis
\usepackage[style=numeric,backend=biber]{biblatex}
% Quellendatenbank
\addbibresource{lit.bib}
\addbibresource{programme.bib}

% Hyperlinks im Dokument
\usepackage[
  unicode,
  pdfusetitle,    % Titel, Autoren und Datum als PDF-Attribute
  pdfcreator={},  % PDF-Attribute säubern
  pdfproducer={}, % "
]{hyperref}
% erweiterte Bookmarks im PDF
\usepackage{bookmark}

% Trennung von Wörtern mit Strichen
\usepackage[shortcuts]{extdash}

\author{
  Helena Nawrath
  \texorpdfstring{
    \\
    \href{mailto:helena.nawrath@tu-dortmund.de}{helena.nawrath@tu-dortmund.de}
  }{}%
  \texorpdfstring{\and}{, }
  Carl Arne Thomann
  \texorpdfstring{
    \\
    \href{mailto:arnethomann@me.com}{arnethomann@me.com}
  }{}
}
\publishers{TU Dortmund – Fakultät Physik}


\subject{201}
\title{Das Dulong-Petitsche Gesetz}
\date{
  Durchführung: 18. November 2014
  \hspace{3em}
  Abgabe: 25. November 2014
}
\begin{document}

\maketitle
\thispagestyle{empty}
\newpage

\section*{Ziel}
Ziel des Versuchs ist es, die Fourier-Transformation kennenzulernen. 
Hierzu wird zum Einen eine bekannte periodische Funktion durch Fourier-Transformation in die Elementarschwingung zerlegt und 
zum Anderen eine periodische Funktion aus Elementarschwingungen gebildet.
\section{Theorie}
\label{sec:theorie}
\subsection{Fourier-Reihenentwicklung, harmonische Analyse}
\label{sec:theorie1}
Für eine T-periodische Funktion $F$ der Zeit gilt
\begin{equation}
	F(t) = F(t+\mathup{T}) \qquad\forall t, 
\end{equation}
für eine Q-periodische Funktion $G$ des Ortes gilt
\begin{equation}
	G(x) = G(x+Q) \qquad\forall x. 
\end{equation}
Bis Abschnitt \ref{sec:theorie3} werden nur solche periodischen Funktionen betrachtet.
Nach dem Fourier'schen Theorem lassen sich periodische Funktionen, etwa Wellenfunktionen, als Linearkombination aus den Elementarschwingungen
\begin{alignat}{3}
	a_\text{n}\text{sin}\Bigl(\frac{2π}{\mathup{T}}t\Bigr) &\qquad\text{und} &&\qquad b_\text{n}\text{cos}\Bigl(\frac{2π}{\mathup{T}}t\Bigr)
\end{alignat}
zusammenfügen.
Das bedeutet, dass die Reihe
\begin{equation}
	\frac{a_0}{2}+\sum_{n=1}^\infty \Bigl(a_\text{n}\text{sin}(\underbrace{n\frac{2π}{\mathup{T}}}_{\omega}t) 
	+ b_\text{n}\text{cos}(\underbrace{n\frac{2π}{\mathup{T}}}_{\omega}t) \Bigr)
	\label{eq:reihe}
\end{equation}
mit $\omega=2\mathup{\pi}\cdot f$
eine T-periodische Funktion $F$ der Zeit darstellt, sofern die Reihe konvergent ist.
Gleichmäßige Konvergenz der Fourier-Reihe ist gegeben, wenn die Funktion $F$ auf ihrem Definitionsbereich stetig ist.

Für die Koeffizienten in \ref{eq:reihe} gilt
\begin{subequations}
\begin{equation}
	a_\text{n} = \frac{2}{\text{T}}\int_0^\text{T} F(t)\text{cos}(2n\mathup{\pi}t)dt
	\label{eq:koeff1}
\end{equation}
\begin{equation}
	b_\text{n} = \frac{2}{\text{T}}\int_0^\text{T} F(t)\text{sin}(2n\mathup{\pi}t)dt.
	\label{eq:koeff2}
\end{equation}
\label{eq:koeff}
\end{subequations}
Die Schwingung wird für $n=0$ Grundschwingung und die Schwingungen für $n>1$ Oberschwingungen bezeichnet.
Für gerade Funktionen $F$, also mit $F(x)=F(-x)$, sind alle $b_\text{n}=0$; 
analog sind für ungerade Funktionen $F$, also mit $-F(x)=F(-x)$, alle $a_\text{n}=0$.

Durch diese Formeln ist allgemein das Beschreiben einer T-periodischen Funktion als konvergente Fourier-Reihe möglich.
Dies wird als harmonische Analyse oder Fourier-Analyse bezeichnet.
\subsection{Linienspektrum der Frequenzen, Spektralanalyse}
\label{sec:theorie2}
Werden die Fourier-Koeffizienten in Gleichungen \eqref{eq:koeff} als Funktionen der Frequenzen $f$ aufgetragen, so ergeben sich Frequenzspektren. 
In diesen ist abgezeichnet, aus welchen Elementarschwingungen die Funktion $F$ besteht und welchen Wert die Fourier-Koeffizienten \eqref{eq:koeff} aufweisen.
\begin{figure}
	\centering
	\includegraphics[width=0.4\textwidth]{Bilder/Linienspektrum.png}
	\caption{Beispiel eines Frequenzspektrums bei konvergenter Fourier-Reihe. \cite{V351}}
	\label{fig:analyse}
\end{figure}
Bei konvergenten Fourier-Reihen \eqref{eq:reihe} geht die Höhe der diskreten Linien für $f\xrightarrow{}\infty$ gegen Null.

\subsection{Nicht-periodische Funktionen, Fourier-Transformation}
\label{sec:theorie3}
Nicht-periodische Funktionen $F$ zeigen bei Spektralanalyse \ref{fig:analyse} ein kontinuierliches Spektrum, weiter gilt nicht $a_\text{n},b_\text{n}\xrightarrow{f\to\infty}0$ im Allgemeinen.
Für diese Funktionen ohne konvergenten Fourier-Reihen wird eine Fourier-Transformation angewandt.

Mit der Fourier-Transformierten $\tilde{F}(\omega)$ und der Funktion $F(t)$ ist die Fourier-Transformation als uneigentliches Integral beschreibbar.
Es gilt
\begin{equation}
	\tilde{F}(\omega)=\int_{-\infty}^{\infty}F(t)\cdot e^{i\omega t} \mathup{d}t
	\label{eq:fourier}
\end{equation}
Die von der Frequenz abhängigen Funktion $\tilde F(\omega)$ beschreibt das in Abschnitt \ref{sec:theorie2} gezeigte Frequenzspektrum. 
Für reale Systeme kann nicht in den geforderten Zeitgrenzenen $-\infty$ und $\infty$ integriert werden; 
durch vorzeitiges Abbrechen der Integration kommt es zu Abweichungen.
Diese Abweichungen bestehen beispielsweise darin, dass eventuell erwartete diskrete Linien (vgl. Abb. \ref{fig:analyse}) als Peaks mit endlicher Breite dargestellt werden und es neben den erwarteten Maxima auch zu Nebenmaxima kommt.

Die Umkehrformel der Fourier-Transformation ist gegeben durch
\begin{equation}
	F(t)=\frac{1}{2\pi}\int_{-\infty}^{\infty}\tilde{F}(\omega)\cdot e^{-i\omega t} \mathup{d}\omega
	\label{eq:reiruof}
\end{equation}
\section{Durchführung}
\label{sec:Durchfuehrung}
\subsection{Fourier-Analyse}
Es werden periodische elektrische Schwingungen in ihre Fourier-Komponenten zerlegt und die gemessenen Fourier-Koeffizienten mit den nach Fourierschen Theorem erwarteten Werten nach Gleichung \ref{eq:koeff} verglichen.

Es wird ein Signal mit den Kenndaten

b) Man setze einfache periodische Schwingungen aus ihren Fourier-Komponenten zusammen und dokumentiere die Ergebnisse durch einen Ausdruck des zeitlichen Verlaufs der Summenspannung.
\section{Auswertung}
\label{sec:Auswertung}
\subsection{Temperaturverläufe}
%Tabelle der gemessenen Temperaturen 
\begin{table}
	\centering
	\sisetup{table-format=2.3}
	\begin{tabular}{S[table-format=1.2] S[table-format=1.3] S[table-format=1.3] }
	\toprule
	\multicolumn{1}{c}{Zeit} & {Temperaturen} \\
	{$t/\:\si{\min}$} & {$T_1/\:\si{\kelvin}$} & {${T_2}/\:\si{\kelvin}$} \\
	\midrule

 0 & 294.45 & 294.45 \\
 1 & 295.35 & 294.45 \\
 2 & 296.15 & 294.35 \\
 3 & 297.45 & 293.45 \\
 4 & 299.05 & 292.05 \\
 5 & 300.85 & 290.25 \\
 6 & 302.95 & 288.25 \\
 7 & 304.85 & 286.45 \\
 8 & 306.85 & 284.65 \\
 9 & 308.65 & 282.85 \\
10 & 310.55 & 281.15 \\
11 & 312.25 & 279.45 \\
12 & 314.05 & 277.75 \\
13 & 315.65 & 276.35 \\
14 & 317.35 & 274.95 \\
15 & 318.85 & 273.95 \\
16 & 320.35 & 273.35 \\
17 & 321.75 & 272.85 \\
18 & 322.95 & 272.45 \\
19 & 324.15 & 272.05 \\
	\bottomrule
	\end{tabular}
	\caption{Zeitabhängige Messung der Temperaturen $T_1$ und $T_2$.}
	\label{tab:Temperaturverlauf}
\end{table}

Die gemessenen Temperaturen $T_1$ (rot) und $T_2$ (blau) der Reservoire werden gegen die Zeit $t$ aufgetragen, um einen ersten Eindruck des Temperaturverlaufes innerhalb der Reservoire zu gewinnen. Dabei ist Reservoir $R_1$ das Behältnis, welches die Wärmemenge $\mathup{d}Q_1$ aufnimmt und sich dabei erhitzt; $R_2$ (blau) bezeichnet das kälter werdende Reservoir.  Werden die Verläufe in einem gemeinsamen Diagramm dargestellt, so lassen sich diese untereinander vergleichen. 
\newpage
\begin{figure}
\includegraphics[width=\textwidth]{Bilder/Temperaturverlauf.pdf}
	\caption{Entwicklung der Wassertemperatur in den Reservoiren $\mathup{R_1}$ und $\mathup{R_2}$.}
	\label{fig:temperaturverlauf}
\end{figure}


Entgegen der Anweisung der Anleitung werden die Verläufe nicht durch eine nicht - lineare Ausgleichsrechnung mit
\begin{equation}
	T_i(t)=A_i t² + B_i t + C_i , i=1,2
	\label{eq:t-verlauf_Grad2}
\end{equation}
 -- einem Polynom zweiten Grades mit den Konstanten $A$, $B$ und $C$ -- genähert. Trotz zweiter Ordnung erscheint der Fit annähernd linear. Da schon anhand der Messwerte zu erkennen ist, dass diese einen Wendepunkt aufweisen wird ein Polynom dritten Grades benutzt; die Messwerte liegen deutlich weniger von der Näherung entfernt ( vgl. Abbildung 2, 3).

\begin{equation}
	T_i(t)=A_i t³ + B_i t² + C_i t + D_i , i=1,2
	\label{eq:t-verlauf_Grad3}
\end{equation}
\newpage
\begin{figure}
\includegraphics[width=\textwidth]{Bilder/Temperaturfit_Grad2.pdf}
	\caption{Annäherung der Kurven durch ein Polynom zweiter Ordnung.}
\end{figure}

\begin{figure}
\includegraphics[width=\textwidth]{Bilder/Temperaturfit.pdf}
	\caption{Annäherung der Kurven durch ein Polynom dritter Ordnung.}
\end{figure}
\newpage
Es ergeben sich für $T_1(t)$ und $T_2(t)$ die Koeffizienten 
\begin{equation}
\begin{split}
A_1&=(-17.1705895195\pm1.7899858311)10⁻⁹\si{\kelvin\per{\second}³}\\
B_1&=(28.2367366738\pm3.10787542417)10⁻⁶\si{\kelvin\per{\second}²}\\
C_1&=(0.0162347592968\pm0.00150160292315)\si{\kelvin\per{\second}}\\
D_1&=(294.111603517\pm0.19259208147)\si{\kelvin}
\end{split}
\end{equation}
und
\begin{equation}
\begin{split}
A_2&=(33.855559517\pm2.46581851787)10⁻⁹\si{\kelvin\per{\second}³}\\
B_2&=(-52.9611716421\pm4.28129424103)10⁻⁶\si{\kelvin\per{\second}²}\\
C_2&=(-0.00325557626607\pm0.00206855306032)\si{\kelvin\per{\second}}\\
D_2&=(294.977609839\pm0.265307963098)\si{\kelvin}.
\end{split}
\end{equation}

Um den Differentialqutionenten $\frac{\mathup{d}T_i}{\mathup{d}t}$ mit $i=1,2$ für verschiedene Zeiten $t_k$ mit $k=1,...,4$ bestimmen zu können, wird die Funktion $T_i(t)$ nach der Zeit $t$ abgeleitet:

\begin{equation}
\frac{\mathup{d}T_i}{\mathup{d}t}= 3A_it²+2B_it+C_i.
\label{ableitung}
\end{equation}

\begin{table}
	\centering
	
	\begin{tabular}{S S S}
	\toprule
	\multicolumn{1}{c}{Zeit} & \multicolumn{2}{c}{Differentialquotienten} \\
	{$t/\:\si{\second}$} & {$\frac{\mathup{d}T_1}{\mathup{d}t}/\:\si{\kelvin{\per\second}}$} & {$\frac{\mathup{d}T_2}{\mathup{d}t}/\:\si{\kelvin{\per\second}}$}\\
	\midrule
 120 & 0.022 $\pm$\:\:46.222   & 0.022 $\pm$\:\:63.674  \\
 480 & 0.032 $\pm$ 184.888   & 0.032 $\pm$ 254.696  \\
 840 & 0.027 $\pm$ 323.555   & 0.027 $\pm$ 445.717  \\
1080 & 0.017 $\pm$ 415.999   & 0.017 $\pm$ 573.065  \\
	\bottomrule
	\end{tabular}
	\caption{Die Differentialqutienten von $T_1$ und $T_2$ zu vier verschiedenen Zeiten $t_k$, berechnet nach Gleichung \eqref{ableitung}.}
	\label{tab:differentialquotienten}
\end{table}

%%%%%%%%%%%%%%%%%%%%%%%%%%%%%%%%%%%%%%%%%%%%%%%%%FEHLERRECHNUNG NACH GAUSS:FORMEL FEHLT NOCH!!!!!!!!!!!!!!!!!


\subsection{Bestimmung der Güteziffer}
Die reale Güteziffer $\nu$ wird mit Hilfe der Messreihe $T_1$ berechnet, indem über den Differenzenquotienten $\frac{\Delta{T_1}}{\Delta{t}}$ die Wärmemenge $\Delta{Q_1}$, welche im Zeitintervall $\Delta{t}$ dem ersten Reservoir zugeführt wird, über die Beziehung 
%das vielleicht noch in die Theorie einbringen??
\begin{equation}
\frac{\Delta{Q_1}}{{\Delta{t}}}=(m_1c_\mathup{w}+m_kc_\mathup{k})\frac{\Delta{T_1}}{{\Delta{t}}}
\label{waermemenge/zeitintervall}
\end{equation}
bestimmt wird.




\section{Diskussion}
\label{sec:Diskussion}
\subsection{}
\subsection{Diskussion des horizontalen Erdmagnetfeldes}
Die Hozizontalkomponente des Erdmagnetfeldes wird für Deutschland mit durchschnittlich \SI{20e-6}{\tesla} \cite{lausitz} angenommen.

\subsection{}
%\nocite{Versuchsnummer}
\printbibliography
\noindent Die verwendeten Plots wurden mit \textit{matplotlib}\cite{matplotlib} und die Grafiken mit \textit{GIMP}\cite{gimp} erstellt sowie die Berechnungen mit Python-\textit{Python-Numpy}, \cite{numpy}, \textit{Python-Scipy}\cite{scipy} und \textit{Python-uncertainties}\cite{uncertainties} durchgeführt.
\end{document}
