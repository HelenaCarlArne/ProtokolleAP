\newpage
\section{Auswertung}
\label{sec:Auswertung}

\subsection{Abstimmen der Systemparameter}
\begin{table}[ht]
	\centering
	\begin{tabular}{ccc}
	\toprule
	{Kapazität $C_\mathup{1,2}$}&{Kapazität der Spule $C_\mathup{sp}$}&{Induktivität der Spule $L$}\\
	{$\si{\pico\farad}$}&{$\si{\pico\farad}$}&{$\si{\milli\henry}$}\\
	\midrule
		798\pm2 &37\pm1 &31.90\pm0.05\\
	\bottomrule
	\end{tabular}
	\caption{Die festen Parameter des gekoppelten Systems nach Abbildung \ref{fig:schaltung}!!.}
\end{table}
Der Kopplungskondensator $C_\mathup{K}$ wird überbrückt und somit effektiv nur der linke Schwingkreis angeregt.
Die Frequenz des Generators, bei welchem die Widerstandsspannung $U_\mathup{R,1}(t)$ maximal wird, ist die Resonanzfrequenz $f_\text{Res}$ des Systems. 
Diese wurde mit $f_\text{Res} = \SI{30.74}{\kilo\hertz}$
grob abgeschätzt.\\
Der Phasenwinkel $\phi$ zwischen Erregerspannung $U(t)$ und der Widerstandsspannung $U_\mathup{R,2}(t)$ ist gleich Null, sofern die Erregerspannung $U(t)$ die Resonanzfrequenz $f_\text{Res}$ aufweist.
Beim Auftragen der Widerstandsspannung $U_\mathup{R,2}(t)$ gegen die Erregerspannung $U(t)$  wird eine Lissajou-Figur sichtbar.
Für den Phasenwinkel $\phi=0$ ist diese Figur nicht-elliptisch, gerade und verläuft im ersten und dritten Quadranten des Oszilloskopes.
Die  Resonanzfrequenz $f_\text{Res}$ wird dadurch auf den Wert
\begin{equation}
	f_\text{Res} = \SI{30.71}{\kilo\hertz}
\end{equation} präzisiert.
Der nach Gleichung \ref{eq:resonanz}!! gegebene Erwartungswert für die Resonanzfrequenz ist
\begin{equation}
	f_\text{Res, Theorie} = \sqrt{\frac{1}{\mathup{LC}}-\frac{\mathup{R^2}}{4\mathup{L^2}}} = \SI{30.84\pm0.05}{\kilo\hertz}.
\end{equation}

Der zweite Schwingkreis wird durch den variablen Kondensator $C_2$ an den ersten Schwingkreis angeglichen.

\subsection{Untersuchung der Schwebung}
\label{sec:Auswertung1}
\begin{table}[ht]
	\centering
	\begin{tabular}{cccc}
	\toprule
	{Kopplungskapazität}&\multicolumn{3}{c}{Frequenzverhältnis}\\
	{$C_\mathup{Kopplung}$}&{$\alpha$}&{$\alpha_\text{Theorie}$}&{$\mathup{\Delta}\alpha$}\\
	{$\si{\nano\farad}$}&{1}&{1}&{\%}\\
	\midrule
		9.99& 14&	14.2\pm0.14		&1.25\\
		8.00& 11&	11.6\pm0.11		&5.1\\
		6.47&  9&	9.6\pm0.09 		&6.11\\
		5.02&  7&	7.6\pm0.07 		&9.18\\
		4.00&  6&	6.3\pm0.06 		&4.97\\
		3.00&  5&	5.0\pm0.04 		&-0.48\\
		2.03&  4&	3.7\pm0.03 		&-7.83\\
		1.01&  2&	2.3\pm0.02 		&15.43\\
	\bottomrule
	\end{tabular}
	\caption{Die Frequenzverhältnisse $\alpha$ in Abhängigkeit von der Kopplungskapazität $C_\mathup{K}$}
	\label{tab:verhaeltnis}
\end{table}
Die Widerstandsspannung $U_\mathup{R,2}(t)$ ist ein Maß für den Strom $I_\mathup{2}(t)$, es besteht nach Ohmschen Gesetz der Zusammenhang $U(t)=\mathup{R}\cdot I(t)$ bei konstantem Widerstand R.
Bei dem Auftragen der Widerstandsspannung $U_\mathup{R,2}(t)$ gegen die Zeit wird gemäß der Theorie \ref{sec:theorie} eine Schwebung sichtbar.
(Helena) Um das Verhältnis der Schwebungs- und Schwingungsfrequenz zu bestimmen, wird innerhalb einer Schwebungsperiode die Anzahl der Schwingungsmaxima abgezählt, der Erwartungswert ist aus Abschnitt \ref{sec:theorie} gegeben. 
Es gilt
\begin{alignat}{3}
	\alpha=\frac{n_\text{Max.Osz.}}{n_\text{Max.Schweb.}} &\quad\text{und} &&\quad\alpha_\text{Theorie}=\frac{f_++f_−}{2(f_+−f_−)},
\end{alignat}
es ergibt sich die Tabelle \ref{tab:verhaeltnis}.

\subsection{Untersuchung der Fundamentallösungen}
\label{sec:Auswertung2}
\begin{table}[ht]
	\centering
	\begin{tabular}{ccccccc}
	\toprule
	{Kopplungskapazität}&\multicolumn{4}{c}{Fundamentalfrequenz}&\multicolumn{2}{c}{Abweichung}\\
	{$C_\mathup{Kopplung}$}&{$f_\mathup{+}$}&{$f_\mathup{-}$}&{$f_\mathup{+,Theorie}$}&{$f_\mathup{-,Theorie}$}&$\mathup{\Delta}f_\mathup{+}$&$\mathup{\Delta}f_\mathup{+}$\\
	{$\si{\nano\farad}$}&{$\si{\kilo\hertz}$}&{$\si{\kilo\hertz}$}&{$\si{\kilo\hertz}$}&{$\si{\kilo\hertz}$}&{$\%$}&{$\%$}\\
	\midrule
		9.99	&30.56	&32.85	 &30.84\pm0.05	&33.09\pm0.05 	&0.90 	&0.73\\
		8.00	&30.57	&33.38	 &30.84\pm0.05	&33.63\pm0.06 	&0.87 	&0.73\\
		6.47	&30.57	&33.98	 &30.84\pm0.05	&34.25\pm0.06 	&0.87 	&0.77\\
		5.02	&30.57	&34.88	 &30.84\pm0.05	&35.16\pm0.06 	&0.87 	&0.78\\
		4.00	&30.57	&35.86	 &30.84\pm0.05	&36.16\pm0.07 	&0.87 	&0.82\\
		3.00	&30.57	&37.40	 &30.84\pm0.05	&37.73\pm0.08 	&0.87 	&0.87\\
		2.03	&30.58	&40.12	 &30.84\pm0.05	&40.5 \pm0.1	&0.84 	&0.97\\
		1.01	&30.58	&47.23	 &30.84\pm0.05	&47.9 \pm0.15	&0.84 	&1.37\\
	\bottomrule
	\end{tabular}
	\caption{Die Frequenzen der Fundamentallösungen in Abhängigkeit von der Kopplungskapazität $C_\mathup{K}$} 
	\label{tab:fundament}
\end{table}
Bei der Untersuchung der Fundamentallösung wird analog zu \ref{sec:Auswerung1} die Lissajou-Figur betrachtet, 
die durch Auftrag der Widerstandsspannung $U_\mathup{R,2}(t)$ gegen die Generatorspannung $U(t)$ angezeigt wird.
Bei einem Phasenwinkel $\phi=0$ bildet sich eine nicht-elliptische, gerade Lissajou-Figur im ersten und dritten Quadranten, bei einem Phasenwinkel $\phi=\pi$ ist diese an der $U_\mathup{R,2}(t)$-Achse gespiegelt.
Es ergeben sich die Frequenzen in Tabelle \ref{tab:fundament}.

Nach Gleichung \eqref{eq:fundament1} und \eqref{eq:fundament2} ergeben sich die theoretischen Werte für die Fundamentalschwingungsfrequenzen in Tabelle \ref{tab:fundamentt}.
\subsection{Untersuchung der Stromkurve $I_2(t)$}
\label{sec:Auswertung2}
\begin{table}[ht]
	\centering
	\begin{tabular}{S[table-format=1.1]S[table-format=1.1]S[table-format=1.2]S[table-format=2.4]S[table-format=2.4]}
	\toprule
	\multicolumn{2}{c}{Zeitkoordinate}&{Kopplungskapazität}&\multicolumn{2}{c}{Frequenz}\\
	{$t_\mathup{+}$}&{$t_\mathup{-}$}&{$C_\mathup{Kopplung}$}&{$f_\mathup{+}$}&{$f_\mathup{-}$}\\
	{$\si{\milli\second}$}&{$\si{\milli\second}$}&{$\si{\nano\farad}$}&{$\si{\kilo\hertz}$}&{$\si{\kilo\hertz}$}\\
	\midrule
		0.3& 	0.9&	9.99&	37.42& 	39.67\\
		0.3&	1.1&	8.00&	37.42&	40.42\\
		0.4&	1.2&	6.47&	37.49&	40.80\\
		0.3&	1.5&	5.02&	37.42&	41.92\\
		0.2&	1.8&	4.00&	37.04&	43.05\\
		0.2&	2.2&	3.00&	37.04&	44.55\\
		0.2&	3.0&	2.03&	37.04&	47.56\\
		0.2&	5.1&	1.01&	37.04&	55.44\\
	\bottomrule
	\end{tabular}
	\caption{Die Zeitkoordinaten und die Frequenzen der Strommaxima in Abhängigkeit von der Kopplungskapazität $C_\text{Kopplung}$.}
	\label{tab:zeitkoord}
\end{table}
Der Generator in Schaltung \ref{fig:schaltung} durchläuft innerhalb von $\SI{20}{\milli\second}$ 
das Frequenzspektrum von \SI{10}{\kilo\hertz} bis \SI{80}{\kilo\hertz}. 
Wird die Widerstandsspannung $U_\mathup{R,2}(t)$ auf einem Oszilloskopen dargestellt, so zeigt sich der Auftrag von der Widerstandsspannung $U_\mathup{R,2}(t)$
gegen die Erregerfrequenz.
Die Zeitkoordinaten der Maxima sind in Tabelle \ref{tab:zeitkoord} dargestellt.

Mit den Anfangs- und Endkoordinaten
$\text{P}_\text{Anfang}=(\SI{-7}{\milli\second}|\SI{10}{\kilo\hertz})$ und $\text{P}_\text{Ende}=(\SI{14,3}{\milli\second}|\SI{80}{\kilo\hertz})$ können die Zeitkoordinaten in Tabelle \ref{tab:zeitkoord} in Frequenzen umgerechnet werden.
Die lineare Regression aus diesen beiden Koordinaten gibt die Umrechnungsformel
\begin{equation}
	f_\pm = \SI{3.7559}{\kilo\hertz\per{\milli\second}}t_\pm+\SI{36.2911}{\kilo\hertz}
\end{equation}
