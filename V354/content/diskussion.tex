\section{Diskussion}
\label{sec:Diskussion}
\subsection{Freie Schwingung}
Im Fall der gedämpften Schwingung fällt die Spannung exponentiell mit der Zeit ab. 
Die einhüllende Exponentialfunktion in Abbildung \ref{tab:extrema} stimmt gut mit den Messwerten überein. 
Der Dämpfungswiderstand $R_\mathup{eff}=(134\pm1)\,\si{\ohm}$ weist eine Abweichung von $99.40\%$ vom Gerätewiderstand $R_1=(67.2\pm0.2)\,\si{\ohm}$ auf.
Vergleich von gemessener und theoretischer Abklingdauer ergibt eine Abweichung von $49.70\%$ bei $T_\mathup{ex}=(251\pm2)\,\si{\micro\second}$ und $T_\mathup{ex,t}=(499\pm3)\,\si\second$. 

Der aperiodische Grenzfall wird bei einem gemessenen Widerstand von $R_\mathup{ap}=4500\,\si\ohm$ realisiert. 
Der theoretisch ermittelte Widerstand $R_\mathup{ap,theo}=5700\,\si\ohm$ weicht um $26.67\%$ ab.\\
Der relativ große Fehler wird durch in der Rechnung nicht betrachtete Leitungs-, Bauteil- und Generatorinnenwiderstände hervorgerufen. 
Unter Berücksichtigung dieser Widerstände würde sich für den Gesamtwiderstand ein größerer Wert ergeben und der Fehler minimiert.

\subsection{Erzwungene Schwingung}
Die Güten $q_1=5,56$ und $q_{1,\mathup{t}}=(4,18\pm0,01)$ unterscheiden sich um $32,92\%$.Der Theoriewert weißt eine relative Abweichung von $0,24\%$.
Zwischen der experimentell bestimmten und theoretisch errechneten Resonanzüberhöhung liegt eine relativ große Abweichung vor, was auf die vielen, in die Rechnung einfließenden, fehlerbehafteten Größen zurückgeführt werden kann. 
Wegen der nicht-konstanten Amplitude $U_0$ lässt sich der Maximalwert der Kondensatorspannung $U_\mathup{C}(t)$ nur ungenau bestimmen. 

Die Schärfe der Resonanz wird über die Resonanzbreite dargestellt. 
Die Abweichung von $7.69\%$ zwischen $\Delta{f}=7\,\si{\kilo\hertz}$ und $\Delta{f_\mathup{t}}=6.5\,\si{\kilo\hertz}$ 
ist  trotz der Tatsache gering, dass $f_+$ und $f_-$ sich durch Auswählen der Messwerte ergeben, zugehörig zu den Bruchteilen der Maximalspannung. 
Das heißt, dass die Genauigkeit durch die Auflösung der Messwertbestimmung und der Genauigkeit der Messwerte gegeben ist. Experimentell ermittelte und theoretisch berechnete Güte unterscheiden sich um $7,82\%$, wobei der theoretische Wert eine geringe relative Abweichung von $0,5\%$ aufweist.
Bei geringen Frequenzen -- gegenüber der Resonanzfrequenz -- tritt ein kleiner Phasenunterschied auf, der mit steigender Frequenz zunimmt. \\
Auch hier stimmt die experimentelle Tatsache mit dem theoretischen $\arctan$-Zusammenhang überein. 
Eine lineare Darstellung im Resonanzbereich gelingt gut; 
an den Rändern ergeben sich geringe Abweichungen durch einen etwas zu groß gewählten Resonanzbereich.
Die auf verschiedene Arten bestimmten Güten liegen zwischen $5,6$ unf $3,25$. Dabei unterscheiden sich die theoretisch errechneten Güten $q_{1,\mathup{t}}=4,18\pm0,01$ und $q_{2,\mathup{t}}=4,00\pm0,02$ nur um $4,5\%$. Die Abweichung des experimentell bestimmten Wertes $q_2=3,71$ von  $q_{1,\mathup{t}}$ und  $q_{2,\mathup{t}}$ liegt in einem annehmbaren Rahmen. $q_3$ und $q_1$ weichen stärker von den Theoriewerten ab.

\subsection{Zusammenfassung}
Das Verhalten von Schwingungen kann elektrisch anhand eines Serienschwingkreises mithilfe eines Oszilloskopen und eines Funktionsgenerators mit hoher Fehlersicherheit untersucht werden.
Die beschriebenen Eigenschaften gelten allgemein für freie und erzwungene Schwingungen, die Erkenntnisse dieses Experimentes sind auf mechanische Systeme übertragbar.
