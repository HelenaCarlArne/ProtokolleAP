\section{Durchführung}
\label{sec:Durchfuehrung}
Das vorherig erwähnte Glasrohr mit Glühdraht, BE und AE befindet sich in einem heizbaren Blechgehäuse, dessen Temperatur $T$ über einen Regler eingestellt und konstant gehalten werden kann.
Das Ablesen der Temperatur erfolgt über ein Thermometer.
Der Heizfaden wird über ein Konstantsapnnungsgerät betrieben, BE und AE über el. Geräte, deren Ausgangsspannung sich zeitproportional ändern kann. 
Die Spannungen können über die Bereiche $0\leq U_\mathup{B} \leq \SI{60}{\volt}$ und $0\leq U_\mathup{A}\leq \SI{11}{\volt}$ variiert werden. Der Auffängerstrom $I_\mathup{A}$ wird über ein Picoamperemeter gemessen.
 Es besteht aus einem Gleichstromverstärker und Aperemeter, das proportional zum Eingangsstrom ausschlägt. 

\begin{figure}
	\includegraphics[width=0.5\textwidth]{Bilder/Aufbau_Detail.pdf}
	\caption{Detaillierter Aufbau des Versuches.}
\end{figure}

\subsection{Franck-Hertz-Kurve}
Um die Franck-Hertz-Kurve aufzeichnen zu können wird ein XY-Schreiber genutzt. 
Auf der X-Achse wird $U_\mathup{B}$ aufgetragen, in Y-Richtung eine Spannung $U$, proportional zum Strom $I_\mathup{A}$. 
Diese Spannung wird vom Picoamperemeter geliefert.
Die Justierung des Schreibers erfolgt über die "zero"-Knöpfe. Damit wird der Nullpunkt in die linke untere Ecke gelegt.
Dabei darf kein Signal an den Eingängen liegen. 
Die Empfindlichkeit der Eingänge wird eingestellt, in dem ein geringes Signal angelegt und langsam gesteigert wird. 
Die Auslenkung des Y-Schreibers sollte maximal sein, wenn $I_\mathup{A}\approx \SI{3}{\nano\ampere}$ erreicht.
 Die X-Komponente sollte vollen Ausschlag zeigen, wenn die Maximalspannung erreicht ist.
 Um die Achse auf Volt zu eichen müssen einige Werte eingetragen werden, die vom Voltmeter abgelesen werden.
Der Hg-Dampfdruck muss wie in THEORIE erwähnt richtig eingestellt werden. 
Dafür wird das Blechgehäuse erhitzt in dem der Stellknopf am Temperaturregler nach rechts gedreht wird, bis ein Ausgangsstrom ein Maximum von ${2,1-2,2}{\si\ampere}$ erreicht wird und $T$ abgelesen werden kann. 
Sobald die gewünschte Temperatur erreicht ist sollte der Knopf  nach links zurückgedreht werden bis der Ausgangsstrom nun ein Minimum von $\SI{1,2}{\ampere}$ erreicht. 
$T$ ist konstant, wenn der Strom zwischen beiden Extremwerten schwingt.
Die Heizleistung des Drahtes sollte so weit heruntergeregelt werden, dass bei $U_\mathup{B}\approx \SI{60}{\volt}$ ein Strom $I_\mathup{A}={1-3}{\si{\nano\ampere}}$ gemessen wird. 
Die Kurve wird bei Temperaturen, die zwischen ${160-200}\si{\celsius}$ liegen, im Bereich von $0\leq U_\mathup{B} \leq \SI{60}{\volt}$ aufgenommen mit $U_\mathup{A}\approx \SI{1}{\volt}$.
\subsection{Energieverteilung der Elektronen}
Der Strom $I_\mathup{A}$ wird in Abhängigkeit von der Bremsspannung $U_\mathup{A}$ aufgezeichnet; $U_\mathup{B}=\SI{11}{\volt}$. 
Die Messung wird bei $T\approx \SI{20}{\degree\celsius},T={140-160}{\si\celsius}$ durchgeführt.
FÜr $T\approx \SI{20}{\celsius}$ wird bei $U_\mathup{A}=0$ ein Strom $I_\mathup{A}=\SI{50}{\volt}$ eingestellt, in dem die Kathodenheizung neu eingeregelt wird.
\subsection{Ionisierungsspannung}
Der Strom $I_\mathup{A}$ wird in Abhängigkeit von $U_\mathup{B}$ aufgenommen bei einer konstanten Anodenspannung von $U_\mathup{A}=\SI{-30}{\volt}$.
