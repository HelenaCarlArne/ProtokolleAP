\section{Diskussion}
\label{sec:Diskussion}

\subsection{Fehleranalyse}
Mithilfe der Kontrolle, gegeben durch Gleichung \eqref{eq:plaus_test}, werden 17 Werte ausgewählt.
Die restlichen 8 Tröpfchen sind wegen starker Abweichung von der Erwartung nicht berücksichtigt worden.
Im Verlauf des Experiments sind Öltröpfchen beobachtet worden, die trotz sorgfältiger Zerstäubung keine Reaktion auf elektrische Felder zeigten. 
Daraus kann geschlossen werden, dass einige Tröpfchen ihre Ladung verloren haben.
Für weitere Versuchsdurchführung nach diesem Aufbau ist darauf zu achten, dass die Ladung nicht abgegeben werden kann.
Mögliche Ursache für die Ladungsabgabe sind eine zu hohe Tröpfchendichte,
im Weiteren kann in einem separaten Versuch der Einfluss untersucht werden, 
den der Ladungszustand des Plattenkondensators im Moment der Einspritzung auf die Ladungsabgabe hat.\\
Außerdem wurden vorwiegend Tröpfchen einer Größe für die Auswertung benützt.
Mit großer Variation in der Tröpfchen-Größe kann der Fehler gering gehalten werden, da eventuelle systematische Fehler, die von der Größe abhängig sind, erkannt und ausgeschlossen werden könnten.

Wegen der hohen Ausscheiderate die geringere Disparität der Teilchen ist mit hoher statischer Unsicherheit der Ergebnissen zu rechnen.

\subsection{Vergleich mit der Literatur}
